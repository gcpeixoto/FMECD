
    




    
\documentclass[11pt]{article}

    
    \usepackage[breakable]{tcolorbox}
    \tcbset{nobeforeafter} % prevents tcolorboxes being placing in paragraphs
    \usepackage{float}
    \floatplacement{figure}{H} % forces figures to be placed at the correct location
    
    \usepackage[T1]{fontenc}
    % Nicer default font (+ math font) than Computer Modern for most use cases
    \usepackage{mathpazo}

    % Basic figure setup, for now with no caption control since it's done
    % automatically by Pandoc (which extracts ![](path) syntax from Markdown).
    \usepackage{graphicx}
    % We will generate all images so they have a width \maxwidth. This means
    % that they will get their normal width if they fit onto the page, but
    % are scaled down if they would overflow the margins.
    \makeatletter
    \def\maxwidth{\ifdim\Gin@nat@width>\linewidth\linewidth
    \else\Gin@nat@width\fi}
    \makeatother
    \let\Oldincludegraphics\includegraphics
    % Set max figure width to be 80% of text width, for now hardcoded.
    \renewcommand{\includegraphics}[1]{\Oldincludegraphics[width=.8\maxwidth]{#1}}
    % Ensure that by default, figures have no caption (until we provide a
    % proper Figure object with a Caption API and a way to capture that
    % in the conversion process - todo).
    \usepackage{caption}
    \DeclareCaptionLabelFormat{nolabel}{}
    \captionsetup{labelformat=nolabel}

    \usepackage{adjustbox} % Used to constrain images to a maximum size 
    \usepackage{xcolor} % Allow colors to be defined
    \usepackage{enumerate} % Needed for markdown enumerations to work
    \usepackage{geometry} % Used to adjust the document margins
    \usepackage{amsmath} % Equations
    \usepackage{amssymb} % Equations
    \usepackage{textcomp} % defines textquotesingle
    % Hack from http://tex.stackexchange.com/a/47451/13684:
    \AtBeginDocument{%
        \def\PYZsq{\textquotesingle}% Upright quotes in Pygmentized code
    }
    \usepackage{upquote} % Upright quotes for verbatim code
    \usepackage{eurosym} % defines \euro
    \usepackage[mathletters]{ucs} % Extended unicode (utf-8) support
    \usepackage[utf8]{inputenc} % Allow utf-8 characters in the tex document
    \usepackage{fancyvrb} % verbatim replacement that allows latex
    \usepackage{grffile} % extends the file name processing of package graphics 
                         % to support a larger range 
    % The hyperref package gives us a pdf with properly built
    % internal navigation ('pdf bookmarks' for the table of contents,
    % internal cross-reference links, web links for URLs, etc.)
    \usepackage{hyperref}
    \usepackage{longtable} % longtable support required by pandoc >1.10
    \usepackage{booktabs}  % table support for pandoc > 1.12.2
    \usepackage[inline]{enumitem} % IRkernel/repr support (it uses the enumerate* environment)
    \usepackage[normalem]{ulem} % ulem is needed to support strikethroughs (\sout)
                                % normalem makes italics be italics, not underlines
    \usepackage{mathrsfs}
    

    
    % Colors for the hyperref package
    \definecolor{urlcolor}{rgb}{0,.145,.698}
    \definecolor{linkcolor}{rgb}{.71,0.21,0.01}
    \definecolor{citecolor}{rgb}{.12,.54,.11}

    % ANSI colors
    \definecolor{ansi-black}{HTML}{3E424D}
    \definecolor{ansi-black-intense}{HTML}{282C36}
    \definecolor{ansi-red}{HTML}{E75C58}
    \definecolor{ansi-red-intense}{HTML}{B22B31}
    \definecolor{ansi-green}{HTML}{00A250}
    \definecolor{ansi-green-intense}{HTML}{007427}
    \definecolor{ansi-yellow}{HTML}{DDB62B}
    \definecolor{ansi-yellow-intense}{HTML}{B27D12}
    \definecolor{ansi-blue}{HTML}{208FFB}
    \definecolor{ansi-blue-intense}{HTML}{0065CA}
    \definecolor{ansi-magenta}{HTML}{D160C4}
    \definecolor{ansi-magenta-intense}{HTML}{A03196}
    \definecolor{ansi-cyan}{HTML}{60C6C8}
    \definecolor{ansi-cyan-intense}{HTML}{258F8F}
    \definecolor{ansi-white}{HTML}{C5C1B4}
    \definecolor{ansi-white-intense}{HTML}{A1A6B2}
    \definecolor{ansi-default-inverse-fg}{HTML}{FFFFFF}
    \definecolor{ansi-default-inverse-bg}{HTML}{000000}

    % commands and environments needed by pandoc snippets
    % extracted from the output of `pandoc -s`
    \providecommand{\tightlist}{%
      \setlength{\itemsep}{0pt}\setlength{\parskip}{0pt}}
    \DefineVerbatimEnvironment{Highlighting}{Verbatim}{commandchars=\\\{\}}
    % Add ',fontsize=\small' for more characters per line
    \newenvironment{Shaded}{}{}
    \newcommand{\KeywordTok}[1]{\textcolor[rgb]{0.00,0.44,0.13}{\textbf{{#1}}}}
    \newcommand{\DataTypeTok}[1]{\textcolor[rgb]{0.56,0.13,0.00}{{#1}}}
    \newcommand{\DecValTok}[1]{\textcolor[rgb]{0.25,0.63,0.44}{{#1}}}
    \newcommand{\BaseNTok}[1]{\textcolor[rgb]{0.25,0.63,0.44}{{#1}}}
    \newcommand{\FloatTok}[1]{\textcolor[rgb]{0.25,0.63,0.44}{{#1}}}
    \newcommand{\CharTok}[1]{\textcolor[rgb]{0.25,0.44,0.63}{{#1}}}
    \newcommand{\StringTok}[1]{\textcolor[rgb]{0.25,0.44,0.63}{{#1}}}
    \newcommand{\CommentTok}[1]{\textcolor[rgb]{0.38,0.63,0.69}{\textit{{#1}}}}
    \newcommand{\OtherTok}[1]{\textcolor[rgb]{0.00,0.44,0.13}{{#1}}}
    \newcommand{\AlertTok}[1]{\textcolor[rgb]{1.00,0.00,0.00}{\textbf{{#1}}}}
    \newcommand{\FunctionTok}[1]{\textcolor[rgb]{0.02,0.16,0.49}{{#1}}}
    \newcommand{\RegionMarkerTok}[1]{{#1}}
    \newcommand{\ErrorTok}[1]{\textcolor[rgb]{1.00,0.00,0.00}{\textbf{{#1}}}}
    \newcommand{\NormalTok}[1]{{#1}}
    
    % Additional commands for more recent versions of Pandoc
    \newcommand{\ConstantTok}[1]{\textcolor[rgb]{0.53,0.00,0.00}{{#1}}}
    \newcommand{\SpecialCharTok}[1]{\textcolor[rgb]{0.25,0.44,0.63}{{#1}}}
    \newcommand{\VerbatimStringTok}[1]{\textcolor[rgb]{0.25,0.44,0.63}{{#1}}}
    \newcommand{\SpecialStringTok}[1]{\textcolor[rgb]{0.73,0.40,0.53}{{#1}}}
    \newcommand{\ImportTok}[1]{{#1}}
    \newcommand{\DocumentationTok}[1]{\textcolor[rgb]{0.73,0.13,0.13}{\textit{{#1}}}}
    \newcommand{\AnnotationTok}[1]{\textcolor[rgb]{0.38,0.63,0.69}{\textbf{\textit{{#1}}}}}
    \newcommand{\CommentVarTok}[1]{\textcolor[rgb]{0.38,0.63,0.69}{\textbf{\textit{{#1}}}}}
    \newcommand{\VariableTok}[1]{\textcolor[rgb]{0.10,0.09,0.49}{{#1}}}
    \newcommand{\ControlFlowTok}[1]{\textcolor[rgb]{0.00,0.44,0.13}{\textbf{{#1}}}}
    \newcommand{\OperatorTok}[1]{\textcolor[rgb]{0.40,0.40,0.40}{{#1}}}
    \newcommand{\BuiltInTok}[1]{{#1}}
    \newcommand{\ExtensionTok}[1]{{#1}}
    \newcommand{\PreprocessorTok}[1]{\textcolor[rgb]{0.74,0.48,0.00}{{#1}}}
    \newcommand{\AttributeTok}[1]{\textcolor[rgb]{0.49,0.56,0.16}{{#1}}}
    \newcommand{\InformationTok}[1]{\textcolor[rgb]{0.38,0.63,0.69}{\textbf{\textit{{#1}}}}}
    \newcommand{\WarningTok}[1]{\textcolor[rgb]{0.38,0.63,0.69}{\textbf{\textit{{#1}}}}}
    
    
    % Define a nice break command that doesn't care if a line doesn't already
    % exist.
    \def\br{\hspace*{\fill} \\* }
    % Math Jax compatibility definitions
    \def\gt{>}
    \def\lt{<}
    \let\Oldtex\TeX
    \let\Oldlatex\LaTeX
    \renewcommand{\TeX}{\textrm{\Oldtex}}
    \renewcommand{\LaTeX}{\textrm{\Oldlatex}}
    % Document parameters
    % Document title
    \title{Aula 4B - Plotagem básica com \emph{matplotlib}}
    % Comment these lines on the .tex converted to see labels
    %\DeclareCaptionFormat{nocaption}{}
    %\captionsetup{format=nocaption,aboveskip=0pt,belowskip=0pt}

% article header template for nbconvert
% jupyter-nbconvert --to latex --template article name.ipynb


\usepackage[brazil]{babel}
\usepackage{microtype}
\usepackage{authblk}
\renewcommand\Authand{ e }
\author[1]{Gustavo Oliveira}
\author[1]{Andrea Rocha}
\affil[1]{Departamento de Computação Científica / UFPB}
\date{Junho de 2020}

    
    
    
    
    
% Pygments definitions
\makeatletter
\def\PY@reset{\let\PY@it=\relax \let\PY@bf=\relax%
    \let\PY@ul=\relax \let\PY@tc=\relax%
    \let\PY@bc=\relax \let\PY@ff=\relax}
\def\PY@tok#1{\csname PY@tok@#1\endcsname}
\def\PY@toks#1+{\ifx\relax#1\empty\else%
    \PY@tok{#1}\expandafter\PY@toks\fi}
\def\PY@do#1{\PY@bc{\PY@tc{\PY@ul{%
    \PY@it{\PY@bf{\PY@ff{#1}}}}}}}
\def\PY#1#2{\PY@reset\PY@toks#1+\relax+\PY@do{#2}}

\expandafter\def\csname PY@tok@w\endcsname{\def\PY@tc##1{\textcolor[rgb]{0.73,0.73,0.73}{##1}}}
\expandafter\def\csname PY@tok@c\endcsname{\let\PY@it=\textit\def\PY@tc##1{\textcolor[rgb]{0.25,0.50,0.50}{##1}}}
\expandafter\def\csname PY@tok@cp\endcsname{\def\PY@tc##1{\textcolor[rgb]{0.74,0.48,0.00}{##1}}}
\expandafter\def\csname PY@tok@k\endcsname{\let\PY@bf=\textbf\def\PY@tc##1{\textcolor[rgb]{0.00,0.50,0.00}{##1}}}
\expandafter\def\csname PY@tok@kp\endcsname{\def\PY@tc##1{\textcolor[rgb]{0.00,0.50,0.00}{##1}}}
\expandafter\def\csname PY@tok@kt\endcsname{\def\PY@tc##1{\textcolor[rgb]{0.69,0.00,0.25}{##1}}}
\expandafter\def\csname PY@tok@o\endcsname{\def\PY@tc##1{\textcolor[rgb]{0.40,0.40,0.40}{##1}}}
\expandafter\def\csname PY@tok@ow\endcsname{\let\PY@bf=\textbf\def\PY@tc##1{\textcolor[rgb]{0.67,0.13,1.00}{##1}}}
\expandafter\def\csname PY@tok@nb\endcsname{\def\PY@tc##1{\textcolor[rgb]{0.00,0.50,0.00}{##1}}}
\expandafter\def\csname PY@tok@nf\endcsname{\def\PY@tc##1{\textcolor[rgb]{0.00,0.00,1.00}{##1}}}
\expandafter\def\csname PY@tok@nc\endcsname{\let\PY@bf=\textbf\def\PY@tc##1{\textcolor[rgb]{0.00,0.00,1.00}{##1}}}
\expandafter\def\csname PY@tok@nn\endcsname{\let\PY@bf=\textbf\def\PY@tc##1{\textcolor[rgb]{0.00,0.00,1.00}{##1}}}
\expandafter\def\csname PY@tok@ne\endcsname{\let\PY@bf=\textbf\def\PY@tc##1{\textcolor[rgb]{0.82,0.25,0.23}{##1}}}
\expandafter\def\csname PY@tok@nv\endcsname{\def\PY@tc##1{\textcolor[rgb]{0.10,0.09,0.49}{##1}}}
\expandafter\def\csname PY@tok@no\endcsname{\def\PY@tc##1{\textcolor[rgb]{0.53,0.00,0.00}{##1}}}
\expandafter\def\csname PY@tok@nl\endcsname{\def\PY@tc##1{\textcolor[rgb]{0.63,0.63,0.00}{##1}}}
\expandafter\def\csname PY@tok@ni\endcsname{\let\PY@bf=\textbf\def\PY@tc##1{\textcolor[rgb]{0.60,0.60,0.60}{##1}}}
\expandafter\def\csname PY@tok@na\endcsname{\def\PY@tc##1{\textcolor[rgb]{0.49,0.56,0.16}{##1}}}
\expandafter\def\csname PY@tok@nt\endcsname{\let\PY@bf=\textbf\def\PY@tc##1{\textcolor[rgb]{0.00,0.50,0.00}{##1}}}
\expandafter\def\csname PY@tok@nd\endcsname{\def\PY@tc##1{\textcolor[rgb]{0.67,0.13,1.00}{##1}}}
\expandafter\def\csname PY@tok@s\endcsname{\def\PY@tc##1{\textcolor[rgb]{0.73,0.13,0.13}{##1}}}
\expandafter\def\csname PY@tok@sd\endcsname{\let\PY@it=\textit\def\PY@tc##1{\textcolor[rgb]{0.73,0.13,0.13}{##1}}}
\expandafter\def\csname PY@tok@si\endcsname{\let\PY@bf=\textbf\def\PY@tc##1{\textcolor[rgb]{0.73,0.40,0.53}{##1}}}
\expandafter\def\csname PY@tok@se\endcsname{\let\PY@bf=\textbf\def\PY@tc##1{\textcolor[rgb]{0.73,0.40,0.13}{##1}}}
\expandafter\def\csname PY@tok@sr\endcsname{\def\PY@tc##1{\textcolor[rgb]{0.73,0.40,0.53}{##1}}}
\expandafter\def\csname PY@tok@ss\endcsname{\def\PY@tc##1{\textcolor[rgb]{0.10,0.09,0.49}{##1}}}
\expandafter\def\csname PY@tok@sx\endcsname{\def\PY@tc##1{\textcolor[rgb]{0.00,0.50,0.00}{##1}}}
\expandafter\def\csname PY@tok@m\endcsname{\def\PY@tc##1{\textcolor[rgb]{0.40,0.40,0.40}{##1}}}
\expandafter\def\csname PY@tok@gh\endcsname{\let\PY@bf=\textbf\def\PY@tc##1{\textcolor[rgb]{0.00,0.00,0.50}{##1}}}
\expandafter\def\csname PY@tok@gu\endcsname{\let\PY@bf=\textbf\def\PY@tc##1{\textcolor[rgb]{0.50,0.00,0.50}{##1}}}
\expandafter\def\csname PY@tok@gd\endcsname{\def\PY@tc##1{\textcolor[rgb]{0.63,0.00,0.00}{##1}}}
\expandafter\def\csname PY@tok@gi\endcsname{\def\PY@tc##1{\textcolor[rgb]{0.00,0.63,0.00}{##1}}}
\expandafter\def\csname PY@tok@gr\endcsname{\def\PY@tc##1{\textcolor[rgb]{1.00,0.00,0.00}{##1}}}
\expandafter\def\csname PY@tok@ge\endcsname{\let\PY@it=\textit}
\expandafter\def\csname PY@tok@gs\endcsname{\let\PY@bf=\textbf}
\expandafter\def\csname PY@tok@gp\endcsname{\let\PY@bf=\textbf\def\PY@tc##1{\textcolor[rgb]{0.00,0.00,0.50}{##1}}}
\expandafter\def\csname PY@tok@go\endcsname{\def\PY@tc##1{\textcolor[rgb]{0.53,0.53,0.53}{##1}}}
\expandafter\def\csname PY@tok@gt\endcsname{\def\PY@tc##1{\textcolor[rgb]{0.00,0.27,0.87}{##1}}}
\expandafter\def\csname PY@tok@err\endcsname{\def\PY@bc##1{\setlength{\fboxsep}{0pt}\fcolorbox[rgb]{1.00,0.00,0.00}{1,1,1}{\strut ##1}}}
\expandafter\def\csname PY@tok@kc\endcsname{\let\PY@bf=\textbf\def\PY@tc##1{\textcolor[rgb]{0.00,0.50,0.00}{##1}}}
\expandafter\def\csname PY@tok@kd\endcsname{\let\PY@bf=\textbf\def\PY@tc##1{\textcolor[rgb]{0.00,0.50,0.00}{##1}}}
\expandafter\def\csname PY@tok@kn\endcsname{\let\PY@bf=\textbf\def\PY@tc##1{\textcolor[rgb]{0.00,0.50,0.00}{##1}}}
\expandafter\def\csname PY@tok@kr\endcsname{\let\PY@bf=\textbf\def\PY@tc##1{\textcolor[rgb]{0.00,0.50,0.00}{##1}}}
\expandafter\def\csname PY@tok@bp\endcsname{\def\PY@tc##1{\textcolor[rgb]{0.00,0.50,0.00}{##1}}}
\expandafter\def\csname PY@tok@fm\endcsname{\def\PY@tc##1{\textcolor[rgb]{0.00,0.00,1.00}{##1}}}
\expandafter\def\csname PY@tok@vc\endcsname{\def\PY@tc##1{\textcolor[rgb]{0.10,0.09,0.49}{##1}}}
\expandafter\def\csname PY@tok@vg\endcsname{\def\PY@tc##1{\textcolor[rgb]{0.10,0.09,0.49}{##1}}}
\expandafter\def\csname PY@tok@vi\endcsname{\def\PY@tc##1{\textcolor[rgb]{0.10,0.09,0.49}{##1}}}
\expandafter\def\csname PY@tok@vm\endcsname{\def\PY@tc##1{\textcolor[rgb]{0.10,0.09,0.49}{##1}}}
\expandafter\def\csname PY@tok@sa\endcsname{\def\PY@tc##1{\textcolor[rgb]{0.73,0.13,0.13}{##1}}}
\expandafter\def\csname PY@tok@sb\endcsname{\def\PY@tc##1{\textcolor[rgb]{0.73,0.13,0.13}{##1}}}
\expandafter\def\csname PY@tok@sc\endcsname{\def\PY@tc##1{\textcolor[rgb]{0.73,0.13,0.13}{##1}}}
\expandafter\def\csname PY@tok@dl\endcsname{\def\PY@tc##1{\textcolor[rgb]{0.73,0.13,0.13}{##1}}}
\expandafter\def\csname PY@tok@s2\endcsname{\def\PY@tc##1{\textcolor[rgb]{0.73,0.13,0.13}{##1}}}
\expandafter\def\csname PY@tok@sh\endcsname{\def\PY@tc##1{\textcolor[rgb]{0.73,0.13,0.13}{##1}}}
\expandafter\def\csname PY@tok@s1\endcsname{\def\PY@tc##1{\textcolor[rgb]{0.73,0.13,0.13}{##1}}}
\expandafter\def\csname PY@tok@mb\endcsname{\def\PY@tc##1{\textcolor[rgb]{0.40,0.40,0.40}{##1}}}
\expandafter\def\csname PY@tok@mf\endcsname{\def\PY@tc##1{\textcolor[rgb]{0.40,0.40,0.40}{##1}}}
\expandafter\def\csname PY@tok@mh\endcsname{\def\PY@tc##1{\textcolor[rgb]{0.40,0.40,0.40}{##1}}}
\expandafter\def\csname PY@tok@mi\endcsname{\def\PY@tc##1{\textcolor[rgb]{0.40,0.40,0.40}{##1}}}
\expandafter\def\csname PY@tok@il\endcsname{\def\PY@tc##1{\textcolor[rgb]{0.40,0.40,0.40}{##1}}}
\expandafter\def\csname PY@tok@mo\endcsname{\def\PY@tc##1{\textcolor[rgb]{0.40,0.40,0.40}{##1}}}
\expandafter\def\csname PY@tok@ch\endcsname{\let\PY@it=\textit\def\PY@tc##1{\textcolor[rgb]{0.25,0.50,0.50}{##1}}}
\expandafter\def\csname PY@tok@cm\endcsname{\let\PY@it=\textit\def\PY@tc##1{\textcolor[rgb]{0.25,0.50,0.50}{##1}}}
\expandafter\def\csname PY@tok@cpf\endcsname{\let\PY@it=\textit\def\PY@tc##1{\textcolor[rgb]{0.25,0.50,0.50}{##1}}}
\expandafter\def\csname PY@tok@c1\endcsname{\let\PY@it=\textit\def\PY@tc##1{\textcolor[rgb]{0.25,0.50,0.50}{##1}}}
\expandafter\def\csname PY@tok@cs\endcsname{\let\PY@it=\textit\def\PY@tc##1{\textcolor[rgb]{0.25,0.50,0.50}{##1}}}

\def\PYZbs{\char`\\}
\def\PYZus{\char`\_}
\def\PYZob{\char`\{}
\def\PYZcb{\char`\}}
\def\PYZca{\char`\^}
\def\PYZam{\char`\&}
\def\PYZlt{\char`\<}
\def\PYZgt{\char`\>}
\def\PYZsh{\char`\#}
\def\PYZpc{\char`\%}
\def\PYZdl{\char`\$}
\def\PYZhy{\char`\-}
\def\PYZsq{\char`\'}
\def\PYZdq{\char`\"}
\def\PYZti{\char`\~}
% for compatibility with earlier versions
\def\PYZat{@}
\def\PYZlb{[}
\def\PYZrb{]}
\makeatother


    % For linebreaks inside Verbatim environment from package fancyvrb. 
    \makeatletter
        \newbox\Wrappedcontinuationbox 
        \newbox\Wrappedvisiblespacebox 
        \newcommand*\Wrappedvisiblespace {\textcolor{red}{\textvisiblespace}} 
        \newcommand*\Wrappedcontinuationsymbol {\textcolor{red}{\llap{\tiny$\m@th\hookrightarrow$}}} 
        \newcommand*\Wrappedcontinuationindent {3ex } 
        \newcommand*\Wrappedafterbreak {\kern\Wrappedcontinuationindent\copy\Wrappedcontinuationbox} 
        % Take advantage of the already applied Pygments mark-up to insert 
        % potential linebreaks for TeX processing. 
        %        {, <, #, %, $, ' and ": go to next line. 
        %        _, }, ^, &, >, - and ~: stay at end of broken line. 
        % Use of \textquotesingle for straight quote. 
        \newcommand*\Wrappedbreaksatspecials {% 
            \def\PYGZus{\discretionary{\char`\_}{\Wrappedafterbreak}{\char`\_}}% 
            \def\PYGZob{\discretionary{}{\Wrappedafterbreak\char`\{}{\char`\{}}% 
            \def\PYGZcb{\discretionary{\char`\}}{\Wrappedafterbreak}{\char`\}}}% 
            \def\PYGZca{\discretionary{\char`\^}{\Wrappedafterbreak}{\char`\^}}% 
            \def\PYGZam{\discretionary{\char`\&}{\Wrappedafterbreak}{\char`\&}}% 
            \def\PYGZlt{\discretionary{}{\Wrappedafterbreak\char`\<}{\char`\<}}% 
            \def\PYGZgt{\discretionary{\char`\>}{\Wrappedafterbreak}{\char`\>}}% 
            \def\PYGZsh{\discretionary{}{\Wrappedafterbreak\char`\#}{\char`\#}}% 
            \def\PYGZpc{\discretionary{}{\Wrappedafterbreak\char`\%}{\char`\%}}% 
            \def\PYGZdl{\discretionary{}{\Wrappedafterbreak\char`\$}{\char`\$}}% 
            \def\PYGZhy{\discretionary{\char`\-}{\Wrappedafterbreak}{\char`\-}}% 
            \def\PYGZsq{\discretionary{}{\Wrappedafterbreak\textquotesingle}{\textquotesingle}}% 
            \def\PYGZdq{\discretionary{}{\Wrappedafterbreak\char`\"}{\char`\"}}% 
            \def\PYGZti{\discretionary{\char`\~}{\Wrappedafterbreak}{\char`\~}}% 
        } 
        % Some characters . , ; ? ! / are not pygmentized. 
        % This macro makes them "active" and they will insert potential linebreaks 
        \newcommand*\Wrappedbreaksatpunct {% 
            \lccode`\~`\.\lowercase{\def~}{\discretionary{\hbox{\char`\.}}{\Wrappedafterbreak}{\hbox{\char`\.}}}% 
            \lccode`\~`\,\lowercase{\def~}{\discretionary{\hbox{\char`\,}}{\Wrappedafterbreak}{\hbox{\char`\,}}}% 
            \lccode`\~`\;\lowercase{\def~}{\discretionary{\hbox{\char`\;}}{\Wrappedafterbreak}{\hbox{\char`\;}}}% 
            \lccode`\~`\:\lowercase{\def~}{\discretionary{\hbox{\char`\:}}{\Wrappedafterbreak}{\hbox{\char`\:}}}% 
            \lccode`\~`\?\lowercase{\def~}{\discretionary{\hbox{\char`\?}}{\Wrappedafterbreak}{\hbox{\char`\?}}}% 
            \lccode`\~`\!\lowercase{\def~}{\discretionary{\hbox{\char`\!}}{\Wrappedafterbreak}{\hbox{\char`\!}}}% 
            \lccode`\~`\/\lowercase{\def~}{\discretionary{\hbox{\char`\/}}{\Wrappedafterbreak}{\hbox{\char`\/}}}% 
            \catcode`\.\active
            \catcode`\,\active 
            \catcode`\;\active
            \catcode`\:\active
            \catcode`\?\active
            \catcode`\!\active
            \catcode`\/\active 
            \lccode`\~`\~ 	
        }
    \makeatother

    \let\OriginalVerbatim=\Verbatim
    \makeatletter
    \renewcommand{\Verbatim}[1][1]{%
        %\parskip\z@skip
        \sbox\Wrappedcontinuationbox {\Wrappedcontinuationsymbol}%
        \sbox\Wrappedvisiblespacebox {\FV@SetupFont\Wrappedvisiblespace}%
        \def\FancyVerbFormatLine ##1{\hsize\linewidth
            \vtop{\raggedright\hyphenpenalty\z@\exhyphenpenalty\z@
                \doublehyphendemerits\z@\finalhyphendemerits\z@
                \strut ##1\strut}%
        }%
        % If the linebreak is at a space, the latter will be displayed as visible
        % space at end of first line, and a continuation symbol starts next line.
        % Stretch/shrink are however usually zero for typewriter font.
        \def\FV@Space {%
            \nobreak\hskip\z@ plus\fontdimen3\font minus\fontdimen4\font
            \discretionary{\copy\Wrappedvisiblespacebox}{\Wrappedafterbreak}
            {\kern\fontdimen2\font}%
        }%
        
        % Allow breaks at special characters using \PYG... macros.
        \Wrappedbreaksatspecials
        % Breaks at punctuation characters . , ; ? ! and / need catcode=\active 	
        \OriginalVerbatim[#1,codes*=\Wrappedbreaksatpunct]%
    }
    \makeatother

    % Exact colors from NB
    \definecolor{incolor}{HTML}{303F9F}
    \definecolor{outcolor}{HTML}{D84315}
    \definecolor{cellborder}{HTML}{CFCFCF}
    \definecolor{cellbackground}{HTML}{F7F7F7}
    
    % prompt
    \newcommand{\prompt}[4]{
        \llap{{\color{#2}[#3]: #4}}\vspace{-1.25em}
    }
    

    
    % Prevent overflowing lines due to hard-to-break entities
    \sloppy 
    % Setup hyperref package
    \hypersetup{
      breaklinks=true,  % so long urls are correctly broken across lines
      colorlinks=true,
      urlcolor=urlcolor,
      linkcolor=linkcolor,
      citecolor=citecolor,
      }
    % Slightly bigger margins than the latex defaults
    
    \geometry{verbose,tmargin=1in,bmargin=1in,lmargin=1in,rmargin=1in}
    
    

    \begin{document}
    
    
    \maketitle
    
    

    
    \hypertarget{plotagem-buxe1sica-com-matplotlib}{%
\section{\texorpdfstring{Plotagem básica com
\emph{matplotlib}}{Plotagem básica com matplotlib}}\label{plotagem-buxe1sica-com-matplotlib}}

    \hypertarget{visualizauxe7uxe3o-de-dados}{%
\subsection{Visualização de dados}\label{visualizauxe7uxe3o-de-dados}}

A visualização de dados é um campo do conhecimento bastante antigo que
foi trazido à mostra muito recentemente com a expansão do ``Big Data''.
Seu principal objetivo é representar dados e informações graficamente
por meio de elementos visuais como tabelas, gráficos, mapas e
infográficos. Diversas ferramentas estão disponíveis para tornar a
interpretação de dados mais clara, compreensível e acessível.

No contexto da análise de dados, a visualização de dados é um componente
fundamental para a criação de relatórios de negócios, painéis de
instrumentos (\emph{dashboards}) e gráficos multidimensionais que são
aplicáveis às mais diversas disciplinas, tais como Economia, Ciência
Política e, principalmente, todo o núcleo de ciências exatas
(Matemática, Estatística e Computação).

Em seu livro \emph{The Visual Display of Quantitative Information},
\href{https://www.edwardtufte.com/tufte/}{{[}Edward Tufte{]}}, conhecido
como o guru do \emph{design} aplicado à visualização de dados, afirma
que, a cada ano, o mundo produz algo entre 900 bilhões e 2 trilhões de
imagens impressas de gráficos. Ele destaca que o \emph{design} de um
gráfico estatístico, por exemplo, é uma matéria universal similar à
Matemática e não está atrelado a características únicas de uma linguagem
particular. Portanto, aprender visualização de dados para comunicar
dados com eficiência é tão importante quanto aprender a Língua
Portuguesa para escrever melhor.

Você pode ver uma lista sugestiva de bons blogues e livros sobre
visualização de dados nas páginas de aprendizagem do software Tableau
\href{https://www.tableau.com/learn/articles/best-data-visualization-blogs}{{[}TabelauBlogs{]}},
\href{https://www.tableau.com/learn/articles/books-about-data-visualization}{{[}TabelauBooks{]}}.

\hypertarget{data-storytelling}{%
\subsection{\texorpdfstring{\emph{Data
storytelling}}{Data storytelling}}\label{data-storytelling}}

\emph{Data Storytelling} é o processo de ``contar histórias através dos
dados''. \href{http://www.storytellingwithdata.com}{{[}Cole Knaflic{]}},
uma engenheira de dados do Google, ao perceber como a quantidade de
informação produzida no mundo às vezes é muito mal lida e comunicada,
escreveu dois \emph{best-sellers} sobre este tema a fim de ajudar
pessoas a comunicarem melhor seus dados e produtos quantitativos. Ela
argumenta em seu livro \emph{Storytelling with Data: A Data
Visualization Guide for Business Professionals} (\emph{Storytelling com
Dados: um Guia Sobre Visualização de Dados Para Profissionais de
Negócios}, na versão em português) que não somos inerentemente bons para
``contar uma história'' através dos dados. Cole mostra com poucas lições
o que devemos aprender para atingir uma comunicação eficiente por meio
da visualização de dados.

\hypertarget{plotagem-matemuxe1tica}{%
\subsection{Plotagem matemática}\label{plotagem-matemuxe1tica}}

\emph{Plotagem} é o termo comumente empregado para o esboço de gráficos
de funções matemáticas via computador. Plotar gráficos é uma das tarefas
que você mais realizará como futuro(a) cientista ou analista de dados.
Nesta aula, nós introduziremos você ao universo da plotagem de gráficos
em duas dimensões e ensinar como você pode visualizar dados facilmente
com a biblioteca \emph{matplotlib}. Daremos uma visão geral
principalmente sobre a plotagem de funções matemáticas utilizando
\emph{arrays} e recursos de computação vetorizada com \emph{numpy} já
aprendidos. Ao longo do curso, você aprenderá a fazer plotagens mais
interessantes de cunho estatístico.

\hypertarget{a-biblioteca-matplotlib}{%
\subsection{\texorpdfstring{A biblioteca
\emph{matplotlib}}{A biblioteca matplotlib}}\label{a-biblioteca-matplotlib}}

\emph{Matplotlib} é a biblioteca Python mais conhecida para plotagem 2D
(bidimensional) de \emph{arrays}. Sua filosofia é simples: criar
plotagens simples com apenas alguns comandos, ou apenas um. John Hunter
\href{https://matplotlib.org/users/history.html}{{[}History{]}},
falecido em 2012, foi o autor desta biblioteca. Em 2008, ele escreveu
que, enquanto buscava uma solução em Python para plotagem 2D, ele
gostaria de ter, entre outras coisas:

\begin{itemize}
\tightlist
\item
  gráficos bonitos com pronta qualidade para publicação;
\item
  capacidade de incorporação em interfaces gráficas para desenvolvimento
  de aplicações;
\item
  um código fácil de entender e de manusear.
\end{itemize}

O \emph{matplotlib} é um código dividido em três partes:

\begin{enumerate}
\def\labelenumi{\arabic{enumi}.}
\tightlist
\item
  A interface \emph{pylab}: um conjunto de funções predefinidas no
  submódulo \texttt{matplotlib.pyplot}.
\item
  O \emph{frontend}: um conjunto de classes responsáveis pela criação de
  figuras, textos, linhas, gráficos etc. No \emph{frontend}, todos os
  elementos gráficos são objetos ainda abstratos.
\item
  O \emph{backend}: um conjunto de renderizadores responsáveis por
  converter os gráficos para dispositivos onde eles podem ser, de fato,
  visualizados. A
  \href{https://pt.wikipedia.org/wiki/Renderização}{{[}renderização{]}}
  é o produto final do processamento digital. Por exemplo, o
  \emph{backend} PS é responsável pela renderização de
  \href{https://www.adobe.com/br/products/postscript.html}{{[}PostScript{]}}.
  Já o \emph{backend} SVG constroi gráficos vetoriais escaláveis
  (\href{https://www.w3.org/Graphics/SVG/}{{[}Scalable Vector
  Graphics{]}}.
\end{enumerate}

Veja o conceito de
\href{https://en.wikipedia.org/wiki/Canvas_(GUI)}{{[}Canvas{]}}.

\hypertarget{sessuxf5es-interativas-do-matplotlib}{%
\subsubsection{\texorpdfstring{Sessões interativas do
\emph{matplotlib}}{Sessões interativas do matplotlib}}\label{sessuxf5es-interativas-do-matplotlib}}

Sessões interativas do \emph{matplotlib} são habilitadas através de um
\href{https://ipython.readthedocs.io/en/stable/interactive/magics.html}{{[}comando
mágico{]}}:

\begin{itemize}
\tightlist
\item
  Em consoles, use \texttt{\%matplotlib};
\item
  No Jupyter notebook, use \texttt{\%matplotlib\ inline}.
\end{itemize}

Lembre que na aula anterior usamos o comando mágico \texttt{\%timeit}
para temporizar operações.

Para usar plenamente o matplotlib nesta aula, vamos usar:

\begin{Shaded}
\begin{Highlighting}[]
\OperatorTok{%}\NormalTok{matplotlib inline}
\ImportTok{from}\NormalTok{ matplotlib }\ImportTok{import}\NormalTok{ pyplot }\ImportTok{as}\NormalTok{ plt}
\end{Highlighting}
\end{Shaded}

A segunda instrução também pode ser feita como

\begin{Shaded}
\begin{Highlighting}[]
\ImportTok{import}\NormalTok{ matplotlib.pyplot }\ImportTok{as}\NormalTok{ plt}
\end{Highlighting}
\end{Shaded}

em que \texttt{plt} é um \emph{alias} já padronizado.

    \begin{tcolorbox}[breakable, size=fbox, boxrule=1pt, pad at break*=1mm,colback=cellbackground, colframe=cellborder]
\prompt{In}{incolor}{1}{\hspace{4pt}}
\begin{Verbatim}[commandchars=\\\{\}]
\PY{c+c1}{\PYZsh{} chamada padrão}
\PY{o}{\PYZpc{}}\PY{k}{matplotlib} inline
\PY{k+kn}{import} \PY{n+nn}{matplotlib}\PY{n+nn}{.}\PY{n+nn}{pyplot} \PY{k}{as} \PY{n+nn}{plt}
\end{Verbatim}
\end{tcolorbox}

    \hypertarget{criauxe7uxe3o-de-plots-simples}{%
\subsection{Criação de plots
simples}\label{criauxe7uxe3o-de-plots-simples}}

Vamos importar o \emph{numpy} para usarmos os benefícios da computação
vetorizada e plotar nossos primeiros exemplos.

    \begin{tcolorbox}[breakable, size=fbox, boxrule=1pt, pad at break*=1mm,colback=cellbackground, colframe=cellborder]
\prompt{In}{incolor}{2}{\hspace{4pt}}
\begin{Verbatim}[commandchars=\\\{\}]
\PY{k+kn}{import} \PY{n+nn}{numpy} \PY{k}{as} \PY{n+nn}{np}

\PY{n}{x} \PY{o}{=} \PY{n}{np}\PY{o}{.}\PY{n}{linspace}\PY{p}{(}\PY{o}{\PYZhy{}}\PY{l+m+mi}{10}\PY{p}{,}\PY{l+m+mi}{10}\PY{p}{,}\PY{l+m+mi}{50}\PY{p}{)}
\PY{n}{y} \PY{o}{=} \PY{n}{x}
\PY{n}{plt}\PY{o}{.}\PY{n}{plot}\PY{p}{(}\PY{n}{x}\PY{p}{,}\PY{n}{y}\PY{p}{)}\PY{p}{;} \PY{c+c1}{\PYZsh{} reta y = x}
\end{Verbatim}
\end{tcolorbox}

    \begin{center}
    \adjustimage{max size={0.9\linewidth}{0.9\paperheight}}{04b-plotagem-matplotlib_files/04b-plotagem-matplotlib_4_0.png}
    \end{center}
    { \hspace*{\fill} \\}
    
    \textbf{Exemplo:} plote o gráfico da parábola \(f(x) = ax^2 + bx + c\)
para valores quaisquer de \(a,b,c\) no intervalo \(-20 \leq x \leq 20\).

    \begin{tcolorbox}[breakable, size=fbox, boxrule=1pt, pad at break*=1mm,colback=cellbackground, colframe=cellborder]
\prompt{In}{incolor}{3}{\hspace{4pt}}
\begin{Verbatim}[commandchars=\\\{\}]
\PY{n}{x} \PY{o}{=} \PY{n}{np}\PY{o}{.}\PY{n}{linspace}\PY{p}{(}\PY{o}{\PYZhy{}}\PY{l+m+mi}{20}\PY{p}{,}\PY{l+m+mi}{21}\PY{p}{,}\PY{l+m+mi}{50}\PY{p}{)}

\PY{n}{a}\PY{p}{,}\PY{n}{b}\PY{p}{,}\PY{n}{c} \PY{o}{=} \PY{l+m+mi}{2}\PY{p}{,}\PY{l+m+mi}{1}\PY{p}{,}\PY{l+m+mi}{4}
\PY{n}{y} \PY{o}{=} \PY{n}{a}\PY{o}{*}\PY{n}{x}\PY{o}{*}\PY{o}{*}\PY{l+m+mi}{2} \PY{o}{+} \PY{n}{b}\PY{o}{*}\PY{n}{x} \PY{o}{+} \PY{n}{c}
\PY{n}{plt}\PY{o}{.}\PY{n}{plot}\PY{p}{(}\PY{n}{x}\PY{p}{,}\PY{n}{y}\PY{p}{)}\PY{p}{;}
\end{Verbatim}
\end{tcolorbox}

    \begin{center}
    \adjustimage{max size={0.9\linewidth}{0.9\paperheight}}{04b-plotagem-matplotlib_files/04b-plotagem-matplotlib_6_0.png}
    \end{center}
    { \hspace*{\fill} \\}
    
    Podemos definir uma função para plotar a parábola:

    \begin{tcolorbox}[breakable, size=fbox, boxrule=1pt, pad at break*=1mm,colback=cellbackground, colframe=cellborder]
\prompt{In}{incolor}{4}{\hspace{4pt}}
\begin{Verbatim}[commandchars=\\\{\}]
\PY{k}{def} \PY{n+nf}{plota\PYZus{}parabola}\PY{p}{(}\PY{n}{a}\PY{p}{,}\PY{n}{b}\PY{p}{,}\PY{n}{c}\PY{p}{)}\PY{p}{:}
    \PY{n}{x} \PY{o}{=} \PY{n}{np}\PY{o}{.}\PY{n}{linspace}\PY{p}{(}\PY{o}{\PYZhy{}}\PY{l+m+mi}{20}\PY{p}{,}\PY{l+m+mi}{21}\PY{p}{,}\PY{l+m+mi}{50}\PY{p}{)}
    \PY{n}{y} \PY{o}{=} \PY{n}{a}\PY{o}{*}\PY{n}{x}\PY{o}{*}\PY{o}{*}\PY{l+m+mi}{2} \PY{o}{+} \PY{n}{b}\PY{o}{*}\PY{n}{x} \PY{o}{+} \PY{n}{c}
    \PY{n}{plt}\PY{o}{.}\PY{n}{plot}\PY{p}{(}\PY{n}{x}\PY{p}{,}\PY{n}{y}\PY{p}{)}
\end{Verbatim}
\end{tcolorbox}

    Agora podemos estudar o que cada coeficiente faz:

    \begin{tcolorbox}[breakable, size=fbox, boxrule=1pt, pad at break*=1mm,colback=cellbackground, colframe=cellborder]
\prompt{In}{incolor}{5}{\hspace{4pt}}
\begin{Verbatim}[commandchars=\\\{\}]
\PY{c+c1}{\PYZsh{} mude o valor de a e considere b = 2, c = 1}

\PY{k}{for} \PY{n}{a} \PY{o+ow}{in} \PY{n}{np}\PY{o}{.}\PY{n}{linspace}\PY{p}{(}\PY{o}{\PYZhy{}}\PY{l+m+mi}{2}\PY{p}{,}\PY{l+m+mi}{3}\PY{p}{,}\PY{l+m+mi}{10}\PY{p}{)}\PY{p}{:}
    \PY{n}{plota\PYZus{}parabola}\PY{p}{(}\PY{n}{a}\PY{p}{,}\PY{l+m+mi}{2}\PY{p}{,}\PY{l+m+mi}{1}\PY{p}{)}
\end{Verbatim}
\end{tcolorbox}

    \begin{center}
    \adjustimage{max size={0.9\linewidth}{0.9\paperheight}}{04b-plotagem-matplotlib_files/04b-plotagem-matplotlib_10_0.png}
    \end{center}
    { \hspace*{\fill} \\}
    
    \begin{tcolorbox}[breakable, size=fbox, boxrule=1pt, pad at break*=1mm,colback=cellbackground, colframe=cellborder]
\prompt{In}{incolor}{6}{\hspace{4pt}}
\begin{Verbatim}[commandchars=\\\{\}]
\PY{c+c1}{\PYZsh{} mude o valor de b e considere a = 2, c = 1}

\PY{k}{for} \PY{n}{b} \PY{o+ow}{in} \PY{n}{np}\PY{o}{.}\PY{n}{linspace}\PY{p}{(}\PY{o}{\PYZhy{}}\PY{l+m+mi}{2}\PY{p}{,}\PY{l+m+mi}{3}\PY{p}{,}\PY{l+m+mi}{10}\PY{p}{)}\PY{p}{:}
    \PY{n}{plota\PYZus{}parabola}\PY{p}{(}\PY{l+m+mi}{2}\PY{p}{,}\PY{n}{b}\PY{p}{,}\PY{l+m+mi}{1}\PY{p}{)}
\end{Verbatim}
\end{tcolorbox}

    \begin{center}
    \adjustimage{max size={0.9\linewidth}{0.9\paperheight}}{04b-plotagem-matplotlib_files/04b-plotagem-matplotlib_11_0.png}
    \end{center}
    { \hspace*{\fill} \\}
    
    \begin{tcolorbox}[breakable, size=fbox, boxrule=1pt, pad at break*=1mm,colback=cellbackground, colframe=cellborder]
\prompt{In}{incolor}{7}{\hspace{4pt}}
\begin{Verbatim}[commandchars=\\\{\}]
\PY{c+c1}{\PYZsh{} mude o valor de c e considere a = 2, b = 1}

\PY{k}{for} \PY{n}{c} \PY{o+ow}{in} \PY{n}{np}\PY{o}{.}\PY{n}{linspace}\PY{p}{(}\PY{o}{\PYZhy{}}\PY{l+m+mi}{2}\PY{p}{,}\PY{l+m+mi}{3}\PY{p}{,}\PY{l+m+mi}{10}\PY{p}{)}\PY{p}{:}
    \PY{n}{plota\PYZus{}parabola}\PY{p}{(}\PY{l+m+mi}{2}\PY{p}{,}\PY{l+m+mi}{1}\PY{p}{,}\PY{n}{c}\PY{p}{)} \PY{c+c1}{\PYZsh{} por que você não vê muitas mudanças?}
\end{Verbatim}
\end{tcolorbox}

    \begin{center}
    \adjustimage{max size={0.9\linewidth}{0.9\paperheight}}{04b-plotagem-matplotlib_files/04b-plotagem-matplotlib_12_0.png}
    \end{center}
    { \hspace*{\fill} \\}
    
    \begin{tcolorbox}[breakable, size=fbox, boxrule=1pt, pad at break*=1mm,colback=cellbackground, colframe=cellborder]
\prompt{In}{incolor}{8}{\hspace{4pt}}
\begin{Verbatim}[commandchars=\\\{\}]
\PY{c+c1}{\PYZsh{} mude o valor de a, b e c }

\PY{n}{valores} \PY{o}{=} \PY{n}{np}\PY{o}{.}\PY{n}{linspace}\PY{p}{(}\PY{o}{\PYZhy{}}\PY{l+m+mi}{2}\PY{p}{,}\PY{l+m+mi}{3}\PY{p}{,}\PY{l+m+mi}{5}\PY{p}{)}
\PY{k}{for} \PY{n}{a} \PY{o+ow}{in} \PY{n}{valores}\PY{p}{:}
    \PY{k}{for} \PY{n}{b} \PY{o+ow}{in} \PY{n}{valores}\PY{p}{:}
        \PY{k}{for} \PY{n}{c} \PY{o+ow}{in} \PY{n}{valores}\PY{p}{:}
            \PY{n}{plota\PYZus{}parabola}\PY{p}{(}\PY{n}{a}\PY{p}{,}\PY{n}{b}\PY{p}{,}\PY{n}{c}\PY{p}{)}
\end{Verbatim}
\end{tcolorbox}

    \begin{center}
    \adjustimage{max size={0.9\linewidth}{0.9\paperheight}}{04b-plotagem-matplotlib_files/04b-plotagem-matplotlib_13_0.png}
    \end{center}
    { \hspace*{\fill} \\}
    
    \textbf{Exemplo:} plote o gráfico da função \(g(t) = a\cos(bt + \pi)\)
para valores quaisquer de \(a\) e \(b\) no intervalo
\(0 \leq t \leq 2\pi\).

    \begin{tcolorbox}[breakable, size=fbox, boxrule=1pt, pad at break*=1mm,colback=cellbackground, colframe=cellborder]
\prompt{In}{incolor}{9}{\hspace{4pt}}
\begin{Verbatim}[commandchars=\\\{\}]
\PY{n}{t} \PY{o}{=} \PY{n}{np}\PY{o}{.}\PY{n}{linspace}\PY{p}{(}\PY{l+m+mi}{0}\PY{p}{,}\PY{l+m+mi}{2}\PY{o}{*}\PY{n}{np}\PY{o}{.}\PY{n}{pi}\PY{p}{,}\PY{l+m+mi}{50}\PY{p}{,}\PY{n}{endpoint}\PY{o}{=}\PY{k+kc}{True}\PY{p}{)}

\PY{n}{a}\PY{p}{,} \PY{n}{b} \PY{o}{=} \PY{l+m+mi}{1}\PY{p}{,} \PY{l+m+mi}{1}
\PY{n}{plt}\PY{o}{.}\PY{n}{plot}\PY{p}{(}\PY{n}{t}\PY{p}{,}\PY{n}{a}\PY{o}{*}\PY{n}{np}\PY{o}{.}\PY{n}{cos}\PY{p}{(}\PY{n}{b}\PY{o}{*}\PY{n}{t} \PY{o}{+} \PY{n}{np}\PY{o}{.}\PY{n}{pi}\PY{p}{)}\PY{p}{)}\PY{p}{;}

\PY{n}{b} \PY{o}{=} \PY{l+m+mi}{2}
\PY{n}{plt}\PY{o}{.}\PY{n}{plot}\PY{p}{(}\PY{n}{t}\PY{p}{,}\PY{n}{a}\PY{o}{*}\PY{n}{np}\PY{o}{.}\PY{n}{cos}\PY{p}{(}\PY{n}{b}\PY{o}{*}\PY{n}{t} \PY{o}{+} \PY{n}{np}\PY{o}{.}\PY{n}{pi}\PY{p}{)}\PY{p}{)}\PY{p}{;}

\PY{n}{b} \PY{o}{=} \PY{l+m+mi}{3}
\PY{n}{plt}\PY{o}{.}\PY{n}{plot}\PY{p}{(}\PY{n}{t}\PY{p}{,}\PY{n}{a}\PY{o}{*}\PY{n}{np}\PY{o}{.}\PY{n}{cos}\PY{p}{(}\PY{n}{b}\PY{o}{*}\PY{n}{t} \PY{o}{+} \PY{n}{np}\PY{o}{.}\PY{n}{pi}\PY{p}{)}\PY{p}{)}\PY{p}{;}
\end{Verbatim}
\end{tcolorbox}

    \begin{center}
    \adjustimage{max size={0.9\linewidth}{0.9\paperheight}}{04b-plotagem-matplotlib_files/04b-plotagem-matplotlib_15_0.png}
    \end{center}
    { \hspace*{\fill} \\}
    
    As cores e marcações no gráfico são todas padronizadas. Vejamos como
alterar tudo isto.

    \hypertarget{alterauxe7uxe3o-de-propriedades-e-estilos-de-linhas}{%
\subsection{Alteração de propriedades e estilos de
linhas}\label{alterauxe7uxe3o-de-propriedades-e-estilos-de-linhas}}

    Altere:

\begin{itemize}
\tightlist
\item
  cores com \texttt{color} ou \texttt{c},
\item
  espessura de linha com \texttt{linewidth} ou \texttt{lw}
\item
  estilo de linha com \texttt{linestyle} ou \texttt{ls}
\item
  tipo de símbolo marcador com \texttt{marker}
\item
  largura de borda do símbolo marcardor com \texttt{markeredgewidth} ou
  \texttt{mew}
\item
  cor de borda do símbolo marcardor com \texttt{markeredgecolor} ou
  \texttt{mec}
\item
  cor de face do símbolo marcardor com \texttt{markerfacecolor} ou
  \texttt{mfc}
\item
  transparência com \texttt{alpha} no intervalo {[}0,1{]}
\end{itemize}

    \begin{tcolorbox}[breakable, size=fbox, boxrule=1pt, pad at break*=1mm,colback=cellbackground, colframe=cellborder]
\prompt{In}{incolor}{10}{\hspace{4pt}}
\begin{Verbatim}[commandchars=\\\{\}]
\PY{n}{g} \PY{o}{=} \PY{k}{lambda} \PY{n}{a}\PY{p}{,}\PY{n}{b}\PY{p}{:} \PY{n}{a}\PY{o}{*}\PY{n}{np}\PY{o}{.}\PY{n}{cos}\PY{p}{(}\PY{n}{b}\PY{o}{*}\PY{n}{t} \PY{o}{+} \PY{n}{np}\PY{o}{.}\PY{n}{pi}\PY{p}{)} \PY{c+c1}{\PYZsh{} assume t anterior}

\PY{c+c1}{\PYZsh{} estude cada exemplo}
\PY{c+c1}{\PYZsh{} a ordem do 3o. argumento em diante pode mudar}
\PY{n}{plt}\PY{o}{.}\PY{n}{plot}\PY{p}{(}\PY{n}{t}\PY{p}{,}\PY{n}{g}\PY{p}{(}\PY{l+m+mi}{1}\PY{p}{,}\PY{l+m+mi}{1}\PY{p}{)}\PY{p}{,}\PY{n}{color}\PY{o}{=}\PY{l+s+s1}{\PYZsq{}}\PY{l+s+s1}{c}\PY{l+s+s1}{\PYZsq{}}\PY{p}{,}\PY{n}{linewidth}\PY{o}{=}\PY{l+m+mi}{5}\PY{p}{,}\PY{n}{linestyle}\PY{o}{=}\PY{l+s+s1}{\PYZsq{}}\PY{l+s+s1}{\PYZhy{}.}\PY{l+s+s1}{\PYZsq{}}\PY{p}{,}\PY{n}{alpha}\PY{o}{=}\PY{l+m+mf}{0.3}\PY{p}{)}
\PY{n}{plt}\PY{o}{.}\PY{n}{plot}\PY{p}{(}\PY{n}{t}\PY{p}{,}\PY{n}{g}\PY{p}{(}\PY{l+m+mi}{1}\PY{p}{,}\PY{l+m+mi}{2}\PY{p}{)}\PY{p}{,}\PY{n}{c}\PY{o}{=}\PY{l+s+s1}{\PYZsq{}}\PY{l+s+s1}{g}\PY{l+s+s1}{\PYZsq{}}\PY{p}{,}\PY{n}{ls}\PY{o}{=}\PY{l+s+s1}{\PYZsq{}}\PY{l+s+s1}{\PYZhy{}}\PY{l+s+s1}{\PYZsq{}}\PY{p}{,}\PY{n}{lw}\PY{o}{=}\PY{l+s+s1}{\PYZsq{}}\PY{l+s+s1}{.7}\PY{l+s+s1}{\PYZsq{}}\PY{p}{,}\PY{n}{marker}\PY{o}{=}\PY{l+s+s1}{\PYZsq{}}\PY{l+s+s1}{s}\PY{l+s+s1}{\PYZsq{}}\PY{p}{,}\PY{n}{mfc}\PY{o}{=}\PY{l+s+s1}{\PYZsq{}}\PY{l+s+s1}{y}\PY{l+s+s1}{\PYZsq{}}\PY{p}{,}\PY{n}{ms}\PY{o}{=}\PY{l+m+mi}{8}\PY{p}{)}
\PY{n}{plt}\PY{o}{.}\PY{n}{plot}\PY{p}{(}\PY{n}{t}\PY{p}{,}\PY{n}{g}\PY{p}{(}\PY{l+m+mi}{1}\PY{p}{,}\PY{l+m+mi}{3}\PY{p}{)}\PY{p}{,}\PY{n}{c}\PY{o}{=}\PY{l+s+s1}{\PYZsq{}}\PY{l+s+s1}{\PYZsh{}e26d5a}\PY{l+s+s1}{\PYZsq{}}\PY{p}{,}\PY{n}{ls}\PY{o}{=}\PY{l+s+s1}{\PYZsq{}}\PY{l+s+s1}{:}\PY{l+s+s1}{\PYZsq{}}\PY{p}{,} \PY{n}{marker}\PY{o}{=}\PY{l+s+s1}{\PYZsq{}}\PY{l+s+s1}{d}\PY{l+s+s1}{\PYZsq{}}\PY{p}{,}\PY{n}{mec}\PY{o}{=}\PY{l+s+s1}{\PYZsq{}}\PY{l+s+s1}{k}\PY{l+s+s1}{\PYZsq{}}\PY{p}{,}\PY{n}{mew}\PY{o}{=}\PY{l+m+mf}{0.9}\PY{p}{)}\PY{p}{;}
\end{Verbatim}
\end{tcolorbox}

    \begin{center}
    \adjustimage{max size={0.9\linewidth}{0.9\paperheight}}{04b-plotagem-matplotlib_files/04b-plotagem-matplotlib_19_0.png}
    \end{center}
    { \hspace*{\fill} \\}
    
    Cores e estilo de linha podem ser especificados de modo reduzido e em
ordens distintas usando um especificador de formato.

    \begin{tcolorbox}[breakable, size=fbox, boxrule=1pt, pad at break*=1mm,colback=cellbackground, colframe=cellborder]
\prompt{In}{incolor}{11}{\hspace{4pt}}
\begin{Verbatim}[commandchars=\\\{\}]
\PY{n}{plt}\PY{o}{.}\PY{n}{plot}\PY{p}{(}\PY{n}{t}\PY{p}{,}\PY{n}{g}\PY{p}{(}\PY{l+m+mi}{1}\PY{p}{,}\PY{l+m+mi}{1}\PY{p}{)}\PY{p}{,}\PY{l+s+s1}{\PYZsq{}}\PY{l+s+s1}{yv}\PY{l+s+s1}{\PYZsq{}}\PY{p}{)} \PY{c+c1}{\PYZsh{} amarelo; triângulo para baixo;}
\PY{n}{plt}\PY{o}{.}\PY{n}{plot}\PY{p}{(}\PY{n}{t}\PY{p}{,}\PY{n}{g}\PY{p}{(}\PY{l+m+mi}{1}\PY{p}{,}\PY{l+m+mi}{2}\PY{p}{)}\PY{p}{,}\PY{l+s+s1}{\PYZsq{}}\PY{l+s+s1}{:c+}\PY{l+s+s1}{\PYZsq{}}\PY{p}{)} \PY{c+c1}{\PYZsh{} pontilhado; ciano; cruz;}
\PY{n}{plt}\PY{o}{.}\PY{n}{plot}\PY{p}{(}\PY{n}{t}\PY{p}{,}\PY{o}{\PYZhy{}}\PY{n}{g}\PY{p}{(}\PY{l+m+mi}{2}\PY{p}{,}\PY{l+m+mi}{2}\PY{p}{)}\PY{p}{,}\PY{l+s+s1}{\PYZsq{}}\PY{l+s+s1}{\PYZgt{}\PYZhy{}.r}\PY{l+s+s1}{\PYZsq{}}\PY{p}{)}\PY{p}{;} \PY{c+c1}{\PYZsh{} triangulo direita; traço\PYZhy{}ponto; vermelho;}
\end{Verbatim}
\end{tcolorbox}

    \begin{center}
    \adjustimage{max size={0.9\linewidth}{0.9\paperheight}}{04b-plotagem-matplotlib_files/04b-plotagem-matplotlib_21_0.png}
    \end{center}
    { \hspace*{\fill} \\}
    
    \hypertarget{plotagem-muxfaltipla}{%
\subsubsection{Plotagem múltipla}\label{plotagem-muxfaltipla}}

    O exemplo acima poderia ser feito como plotagem múltipla em 3 blocos do
tipo (\texttt{x,y,\textquotesingle{}fmt\textquotesingle{})}, onde
\texttt{x} e \texttt{y} são as informações dos eixos coordenados e
\texttt{fmt} é uma string de formatação.

    \begin{tcolorbox}[breakable, size=fbox, boxrule=1pt, pad at break*=1mm,colback=cellbackground, colframe=cellborder]
\prompt{In}{incolor}{12}{\hspace{4pt}}
\begin{Verbatim}[commandchars=\\\{\}]
\PY{n}{plt}\PY{o}{.}\PY{n}{plot}\PY{p}{(}\PY{n}{t}\PY{p}{,}\PY{n}{g}\PY{p}{(}\PY{l+m+mi}{1}\PY{p}{,}\PY{l+m+mi}{1}\PY{p}{)}\PY{p}{,}\PY{l+s+s1}{\PYZsq{}}\PY{l+s+s1}{yv}\PY{l+s+s1}{\PYZsq{}}\PY{p}{,} \PY{n}{t}\PY{p}{,}\PY{n}{g}\PY{p}{(}\PY{l+m+mi}{1}\PY{p}{,}\PY{l+m+mi}{2}\PY{p}{)}\PY{p}{,}\PY{l+s+s1}{\PYZsq{}}\PY{l+s+s1}{:c+}\PY{l+s+s1}{\PYZsq{}}\PY{p}{,} \PY{n}{t}\PY{p}{,}\PY{o}{\PYZhy{}}\PY{n}{g}\PY{p}{(}\PY{l+m+mi}{2}\PY{p}{,}\PY{l+m+mi}{2}\PY{p}{)}\PY{p}{,}\PY{l+s+s1}{\PYZsq{}}\PY{l+s+s1}{\PYZgt{}\PYZhy{}.r}\PY{l+s+s1}{\PYZsq{}}\PY{p}{)}\PY{p}{;} \PY{c+c1}{\PYZsh{} 3 blocos sequenciados}
\end{Verbatim}
\end{tcolorbox}

    \begin{center}
    \adjustimage{max size={0.9\linewidth}{0.9\paperheight}}{04b-plotagem-matplotlib_files/04b-plotagem-matplotlib_24_0.png}
    \end{center}
    { \hspace*{\fill} \\}
    
    Para verificar todas as opções de propriedades e estilos de linhas, veja
\texttt{plt.plot?}.

    \hypertarget{especificauxe7uxe3o-de-figuras}{%
\subsubsection{Especificação de
figuras}\label{especificauxe7uxe3o-de-figuras}}

Use \texttt{plt.figure} para criar um ambiente de figura e altere:

\begin{itemize}
\tightlist
\item
  a largura e altura (em polegadas) com
  \texttt{figsize\ =\ (largura,altura)}. O padrão é (6.4,4.8).
\item
  a resolução (em pontos por polegadas) com \texttt{dpi}. O padrão é
  100.
\item
  a cor de fundo (\emph{background}) com \texttt{facecolor}. O padrão é
  \texttt{w} (branco).
\end{itemize}

    \textbf{Exemplo:} Plote os gráficos de \(h_1(x) = a\sqrt{x}\)
\(h_2(x) = be^{\frac{x}{c}}\) para valores de a,b,c epropriedades acima
livres.

    \begin{tcolorbox}[breakable, size=fbox, boxrule=1pt, pad at break*=1mm,colback=cellbackground, colframe=cellborder]
\prompt{In}{incolor}{13}{\hspace{4pt}}
\begin{Verbatim}[commandchars=\\\{\}]
\PY{n}{x} \PY{o}{=} \PY{n}{np}\PY{o}{.}\PY{n}{linspace}\PY{p}{(}\PY{l+m+mi}{0}\PY{p}{,}\PY{l+m+mi}{10}\PY{p}{,}\PY{l+m+mi}{50}\PY{p}{,}\PY{n}{endpoint}\PY{o}{=}\PY{k+kc}{True}\PY{p}{)}

\PY{n}{h1}\PY{p}{,} \PY{n}{h2} \PY{o}{=} \PY{k}{lambda} \PY{n}{a}\PY{p}{:} \PY{n}{a}\PY{o}{*}\PY{n}{np}\PY{o}{.}\PY{n}{sqrt}\PY{p}{(}\PY{n}{x}\PY{p}{)}\PY{p}{,} \PY{k}{lambda} \PY{n}{b}\PY{p}{,}\PY{n}{c}\PY{p}{:} \PY{n}{b}\PY{o}{*}\PY{n}{np}\PY{o}{.}\PY{n}{exp}\PY{p}{(}\PY{n}{x}\PY{o}{/}\PY{n}{c}\PY{p}{)} 

\PY{n}{plt}\PY{o}{.}\PY{n}{figure}\PY{p}{(}\PY{n}{figsize}\PY{o}{=}\PY{p}{(}\PY{l+m+mi}{8}\PY{p}{,}\PY{l+m+mi}{6}\PY{p}{)}\PY{p}{,} \PY{n}{dpi}\PY{o}{=}\PY{l+m+mi}{100}\PY{p}{,} \PY{n}{facecolor}\PY{o}{=}\PY{l+s+s1}{\PYZsq{}}\PY{l+s+s1}{\PYZsh{}e0eeee}\PY{l+s+s1}{\PYZsq{}}\PY{p}{)}
\PY{n}{plt}\PY{o}{.}\PY{n}{plot}\PY{p}{(}\PY{n}{x}\PY{p}{,}\PY{n}{h1}\PY{p}{(}\PY{o}{.}\PY{l+m+mi}{9}\PY{p}{)}\PY{p}{,}\PY{n}{x}\PY{p}{,}\PY{n}{h2}\PY{p}{(}\PY{l+m+mi}{1}\PY{p}{,}\PY{l+m+mi}{9}\PY{p}{)}\PY{p}{)}\PY{p}{;}
\end{Verbatim}
\end{tcolorbox}

    \begin{center}
    \adjustimage{max size={0.9\linewidth}{0.9\paperheight}}{04b-plotagem-matplotlib_files/04b-plotagem-matplotlib_28_0.png}
    \end{center}
    { \hspace*{\fill} \\}
    
    \hypertarget{alterando-limites-e-marcauxe7uxf5es-de-eixos}{%
\subsubsection{Alterando limites e marcações de
eixos}\label{alterando-limites-e-marcauxe7uxf5es-de-eixos}}

Altere:

\begin{itemize}
\tightlist
\item
  o intervalo do eixo \texttt{x} com \texttt{xlim}
\item
  o intervalo do eixo \texttt{y} com \texttt{ylim}
\item
  as marcações do eixo \texttt{x} com \texttt{xticks}
\item
  as marcações do eixo \texttt{y} com \texttt{yticks}
\end{itemize}

    \begin{tcolorbox}[breakable, size=fbox, boxrule=1pt, pad at break*=1mm,colback=cellbackground, colframe=cellborder]
\prompt{In}{incolor}{14}{\hspace{4pt}}
\begin{Verbatim}[commandchars=\\\{\}]
\PY{n}{plt}\PY{o}{.}\PY{n}{plot}\PY{p}{(}\PY{n}{x}\PY{p}{,}\PY{n}{h1}\PY{p}{(}\PY{o}{.}\PY{l+m+mi}{9}\PY{p}{)}\PY{p}{,}\PY{n}{x}\PY{p}{,}\PY{n}{h2}\PY{p}{(}\PY{l+m+mi}{1}\PY{p}{,}\PY{l+m+mi}{9}\PY{p}{)}\PY{p}{)}\PY{p}{;} \PY{n}{plt}\PY{o}{.}\PY{n}{xlim}\PY{p}{(}\PY{l+m+mf}{1.6}\PY{p}{,}\PY{l+m+mf}{9.2}\PY{p}{)}\PY{p}{;} \PY{n}{plt}\PY{o}{.}\PY{n}{ylim}\PY{p}{(}\PY{l+m+mf}{1.0}\PY{p}{,}\PY{l+m+mf}{2.8}\PY{p}{)}\PY{p}{;}
\end{Verbatim}
\end{tcolorbox}

    \begin{center}
    \adjustimage{max size={0.9\linewidth}{0.9\paperheight}}{04b-plotagem-matplotlib_files/04b-plotagem-matplotlib_30_0.png}
    \end{center}
    { \hspace*{\fill} \\}
    
    \begin{tcolorbox}[breakable, size=fbox, boxrule=1pt, pad at break*=1mm,colback=cellbackground, colframe=cellborder]
\prompt{In}{incolor}{15}{\hspace{4pt}}
\begin{Verbatim}[commandchars=\\\{\}]
\PY{n}{plt}\PY{o}{.}\PY{n}{figure}\PY{p}{(}\PY{n}{figsize}\PY{o}{=}\PY{p}{(}\PY{l+m+mi}{10}\PY{p}{,}\PY{l+m+mi}{8}\PY{p}{)}\PY{p}{)}

\PY{n}{plt}\PY{o}{.}\PY{n}{plot}\PY{p}{(}\PY{n}{t}\PY{p}{,}\PY{n}{g}\PY{p}{(}\PY{l+m+mi}{1}\PY{p}{,}\PY{l+m+mi}{3}\PY{p}{)}\PY{p}{,}\PY{n}{c}\PY{o}{=}\PY{p}{[}\PY{l+m+mf}{0.1}\PY{p}{,}\PY{l+m+mf}{0.4}\PY{p}{,}\PY{l+m+mf}{0.5}\PY{p}{]}\PY{p}{,}\PY{n}{marker}\PY{o}{=}\PY{l+s+s1}{\PYZsq{}}\PY{l+s+s1}{s}\PY{l+s+s1}{\PYZsq{}}\PY{p}{,}\PY{n}{mfc}\PY{o}{=}\PY{l+s+s1}{\PYZsq{}}\PY{l+s+s1}{w}\PY{l+s+s1}{\PYZsq{}}\PY{p}{,}\PY{n}{mew}\PY{o}{=}\PY{l+m+mf}{2.0}\PY{p}{)}\PY{p}{;}
\PY{n}{plt}\PY{o}{.}\PY{n}{plot}\PY{p}{(}\PY{n}{t}\PY{p}{,}\PY{n}{g}\PY{p}{(}\PY{l+m+mf}{1.2}\PY{p}{,}\PY{l+m+mi}{2}\PY{p}{)}\PY{p}{,}\PY{n}{c}\PY{o}{=}\PY{p}{[}\PY{l+m+mf}{1.0}\PY{p}{,}\PY{l+m+mf}{0.5}\PY{p}{,}\PY{l+m+mf}{0.0}\PY{p}{]}\PY{p}{,}\PY{n}{ls}\PY{o}{=}\PY{l+s+s1}{\PYZsq{}}\PY{l+s+s1}{\PYZhy{}\PYZhy{}}\PY{l+s+s1}{\PYZsq{}}\PY{p}{,}\PY{n}{marker}\PY{o}{=}\PY{l+s+s1}{\PYZsq{}}\PY{l+s+s1}{\PYZgt{}}\PY{l+s+s1}{\PYZsq{}}\PY{p}{,}\PY{n}{mfc}\PY{o}{=}\PY{l+s+s1}{\PYZsq{}}\PY{l+s+s1}{c}\PY{l+s+s1}{\PYZsq{}}\PY{p}{,}\PY{n}{mew}\PY{o}{=}\PY{l+m+mf}{1.0}\PY{p}{,}\PY{n}{ms}\PY{o}{=}\PY{l+m+mi}{10}\PY{p}{)}\PY{p}{;}

\PY{n}{plt}\PY{o}{.}\PY{n}{xticks}\PY{p}{(}\PY{p}{[}\PY{l+m+mi}{0}\PY{p}{,} \PY{n}{np}\PY{o}{.}\PY{n}{pi}\PY{o}{/}\PY{l+m+mi}{2}\PY{p}{,}\PY{n}{np}\PY{o}{.}\PY{n}{pi}\PY{p}{,}\PY{l+m+mi}{3}\PY{o}{*}\PY{n}{np}\PY{o}{.}\PY{n}{pi}\PY{o}{/}\PY{l+m+mi}{2}\PY{p}{,}\PY{l+m+mi}{2}\PY{o}{*}\PY{n}{np}\PY{o}{.}\PY{n}{pi}\PY{p}{]}\PY{p}{)}\PY{p}{;} \PY{c+c1}{\PYZsh{} lista de múltiplos de pi}
\PY{n}{plt}\PY{o}{.}\PY{n}{yticks}\PY{p}{(}\PY{p}{[}\PY{o}{\PYZhy{}}\PY{l+m+mi}{1}\PY{p}{,} \PY{l+m+mi}{0}\PY{p}{,} \PY{l+m+mi}{1}\PY{p}{]}\PY{p}{)}\PY{p}{;} \PY{c+c1}{\PYZsh{} 3 valores em y}
\end{Verbatim}
\end{tcolorbox}

    \begin{center}
    \adjustimage{max size={0.9\linewidth}{0.9\paperheight}}{04b-plotagem-matplotlib_files/04b-plotagem-matplotlib_31_0.png}
    \end{center}
    { \hspace*{\fill} \\}
    
    \hypertarget{especificando-texto-de-marcauxe7uxf5es-em-eixos}{%
\subsubsection{Especificando texto de marcações em
eixos}\label{especificando-texto-de-marcauxe7uxf5es-em-eixos}}

Podemos alterar as marcações das \texttt{ticks} passando um texto
indicativo. No caso anterior, seria melhor algo como:

    \begin{tcolorbox}[breakable, size=fbox, boxrule=1pt, pad at break*=1mm,colback=cellbackground, colframe=cellborder]
\prompt{In}{incolor}{16}{\hspace{4pt}}
\begin{Verbatim}[commandchars=\\\{\}]
\PY{n}{plt}\PY{o}{.}\PY{n}{figure}\PY{p}{(}\PY{n}{figsize}\PY{o}{=}\PY{p}{(}\PY{l+m+mi}{10}\PY{p}{,}\PY{l+m+mi}{8}\PY{p}{)}\PY{p}{)}

\PY{n}{plt}\PY{o}{.}\PY{n}{plot}\PY{p}{(}\PY{n}{t}\PY{p}{,}\PY{n}{g}\PY{p}{(}\PY{l+m+mi}{1}\PY{p}{,}\PY{l+m+mi}{3}\PY{p}{)}\PY{p}{,}\PY{n}{c}\PY{o}{=}\PY{p}{[}\PY{l+m+mf}{0.1}\PY{p}{,}\PY{l+m+mf}{0.4}\PY{p}{,}\PY{l+m+mf}{0.5}\PY{p}{]}\PY{p}{,}\PY{n}{marker}\PY{o}{=}\PY{l+s+s1}{\PYZsq{}}\PY{l+s+s1}{s}\PY{l+s+s1}{\PYZsq{}}\PY{p}{,}\PY{n}{mfc}\PY{o}{=}\PY{l+s+s1}{\PYZsq{}}\PY{l+s+s1}{w}\PY{l+s+s1}{\PYZsq{}}\PY{p}{,}\PY{n}{mew}\PY{o}{=}\PY{l+m+mf}{2.0}\PY{p}{)}\PY{p}{;}
\PY{n}{plt}\PY{o}{.}\PY{n}{plot}\PY{p}{(}\PY{n}{t}\PY{p}{,}\PY{n}{g}\PY{p}{(}\PY{l+m+mf}{1.2}\PY{p}{,}\PY{l+m+mi}{2}\PY{p}{)}\PY{p}{,}\PY{n}{c}\PY{o}{=}\PY{p}{[}\PY{l+m+mf}{1.0}\PY{p}{,}\PY{l+m+mf}{0.5}\PY{p}{,}\PY{l+m+mf}{0.0}\PY{p}{]}\PY{p}{,}\PY{n}{ls}\PY{o}{=}\PY{l+s+s1}{\PYZsq{}}\PY{l+s+s1}{\PYZhy{}\PYZhy{}}\PY{l+s+s1}{\PYZsq{}}\PY{p}{,}\PY{n}{marker}\PY{o}{=}\PY{l+s+s1}{\PYZsq{}}\PY{l+s+s1}{\PYZgt{}}\PY{l+s+s1}{\PYZsq{}}\PY{p}{,}\PY{n}{mfc}\PY{o}{=}\PY{l+s+s1}{\PYZsq{}}\PY{l+s+s1}{c}\PY{l+s+s1}{\PYZsq{}}\PY{p}{,}\PY{n}{mew}\PY{o}{=}\PY{l+m+mf}{1.0}\PY{p}{,}\PY{n}{ms}\PY{o}{=}\PY{l+m+mi}{10}\PY{p}{)}\PY{p}{;}

\PY{c+c1}{\PYZsh{} o par de \PYZdl{}...\PYZdl{} formata os números na linguagem TeX}
\PY{n}{plt}\PY{o}{.}\PY{n}{xticks}\PY{p}{(}\PY{p}{[}\PY{l+m+mi}{0}\PY{p}{,} \PY{n}{np}\PY{o}{.}\PY{n}{pi}\PY{o}{/}\PY{l+m+mi}{2}\PY{p}{,}\PY{n}{np}\PY{o}{.}\PY{n}{pi}\PY{p}{,}\PY{l+m+mi}{3}\PY{o}{*}\PY{n}{np}\PY{o}{.}\PY{n}{pi}\PY{o}{/}\PY{l+m+mi}{2}\PY{p}{,}\PY{l+m+mi}{2}\PY{o}{*}\PY{n}{np}\PY{o}{.}\PY{n}{pi}\PY{p}{]}\PY{p}{,} \PY{p}{[}\PY{l+s+s1}{\PYZsq{}}\PY{l+s+s1}{\PYZdl{}0\PYZdl{}}\PY{l+s+s1}{\PYZsq{}}\PY{p}{,}\PY{l+s+s1}{\PYZsq{}}\PY{l+s+s1}{\PYZdl{}}\PY{l+s+s1}{\PYZbs{}}\PY{l+s+s1}{pi/2\PYZdl{}}\PY{l+s+s1}{\PYZsq{}}\PY{p}{,}\PY{l+s+s1}{\PYZsq{}}\PY{l+s+s1}{\PYZdl{}}\PY{l+s+s1}{\PYZbs{}}\PY{l+s+s1}{pi\PYZdl{}}\PY{l+s+s1}{\PYZsq{}}\PY{p}{,}\PY{l+s+s1}{\PYZsq{}}\PY{l+s+s1}{\PYZdl{}3/2}\PY{l+s+s1}{\PYZbs{}}\PY{l+s+s1}{pi\PYZdl{}}\PY{l+s+s1}{\PYZsq{}}\PY{p}{,}\PY{l+s+s1}{\PYZsq{}}\PY{l+s+s1}{\PYZdl{}2}\PY{l+s+s1}{\PYZbs{}}\PY{l+s+s1}{pi\PYZdl{}}\PY{l+s+s1}{\PYZsq{}}\PY{p}{]}\PY{p}{)}\PY{p}{;} 
\PY{n}{plt}\PY{o}{.}\PY{n}{yticks}\PY{p}{(}\PY{p}{[}\PY{o}{\PYZhy{}}\PY{l+m+mi}{1}\PY{p}{,} \PY{l+m+mi}{0}\PY{p}{,} \PY{l+m+mi}{1}\PY{p}{]}\PY{p}{,} \PY{p}{[}\PY{l+s+s1}{\PYZsq{}}\PY{l+s+s1}{\PYZdl{}y = \PYZhy{}1\PYZdl{}}\PY{l+s+s1}{\PYZsq{}}\PY{p}{,} \PY{l+s+s1}{\PYZsq{}}\PY{l+s+s1}{\PYZdl{}y = 0\PYZdl{}}\PY{l+s+s1}{\PYZsq{}}\PY{p}{,} \PY{l+s+s1}{\PYZsq{}}\PY{l+s+s1}{\PYZdl{}y = +1\PYZdl{}}\PY{l+s+s1}{\PYZsq{}}\PY{p}{]}\PY{p}{)}\PY{p}{;} 
\end{Verbatim}
\end{tcolorbox}

    \begin{center}
    \adjustimage{max size={0.9\linewidth}{0.9\paperheight}}{04b-plotagem-matplotlib_files/04b-plotagem-matplotlib_33_0.png}
    \end{center}
    { \hspace*{\fill} \\}
    
    \hypertarget{deslocamento-de-eixos-principais}{%
\subsubsection{Deslocamento de eixos
principais}\label{deslocamento-de-eixos-principais}}

Os eixos principais podem ser movidos para outras posições arbitrárias e
as bordas da área de plotagem desligadas usando \texttt{spine}.

    \begin{tcolorbox}[breakable, size=fbox, boxrule=1pt, pad at break*=1mm,colback=cellbackground, colframe=cellborder]
\prompt{In}{incolor}{17}{\hspace{4pt}}
\begin{Verbatim}[commandchars=\\\{\}]
\PY{c+c1}{\PYZsh{} plotagem da função}
\PY{n}{x} \PY{o}{=} \PY{n}{np}\PY{o}{.}\PY{n}{linspace}\PY{p}{(}\PY{o}{\PYZhy{}}\PY{l+m+mi}{3}\PY{p}{,}\PY{l+m+mi}{3}\PY{p}{)}
\PY{n}{plt}\PY{o}{.}\PY{n}{plot}\PY{p}{(}\PY{n}{x}\PY{p}{,}\PY{n}{x}\PY{o}{*}\PY{o}{*}\PY{l+m+mi}{1}\PY{o}{/}\PY{l+m+mi}{2}\PY{o}{*}\PY{n}{np}\PY{o}{.}\PY{n}{sin}\PY{p}{(}\PY{n}{x}\PY{p}{)}\PY{o}{\PYZhy{}}\PY{l+m+mf}{0.5}\PY{p}{)}\PY{p}{;}
\PY{n}{ax} \PY{o}{=} \PY{n}{plt}\PY{o}{.}\PY{n}{gca}\PY{p}{(}\PY{p}{)}  

\PY{n}{ax}\PY{o}{.}\PY{n}{spines}\PY{p}{[}\PY{l+s+s1}{\PYZsq{}}\PY{l+s+s1}{right}\PY{l+s+s1}{\PYZsq{}}\PY{p}{]}\PY{o}{.}\PY{n}{set\PYZus{}color}\PY{p}{(}\PY{l+s+s1}{\PYZsq{}}\PY{l+s+s1}{none}\PY{l+s+s1}{\PYZsq{}}\PY{p}{)} \PY{c+c1}{\PYZsh{} remove borda direita}
\PY{n}{ax}\PY{o}{.}\PY{n}{spines}\PY{p}{[}\PY{l+s+s1}{\PYZsq{}}\PY{l+s+s1}{top}\PY{l+s+s1}{\PYZsq{}}\PY{p}{]}\PY{o}{.}\PY{n}{set\PYZus{}color}\PY{p}{(}\PY{l+s+s1}{\PYZsq{}}\PY{l+s+s1}{none}\PY{l+s+s1}{\PYZsq{}}\PY{p}{)} \PY{c+c1}{\PYZsh{} remove borda superior}

\PY{n}{ax}\PY{o}{.}\PY{n}{spines}\PY{p}{[}\PY{l+s+s1}{\PYZsq{}}\PY{l+s+s1}{bottom}\PY{l+s+s1}{\PYZsq{}}\PY{p}{]}\PY{o}{.}\PY{n}{set\PYZus{}position}\PY{p}{(}\PY{p}{(}\PY{l+s+s1}{\PYZsq{}}\PY{l+s+s1}{data}\PY{l+s+s1}{\PYZsq{}}\PY{p}{,}\PY{l+m+mi}{0}\PY{p}{)}\PY{p}{)} \PY{c+c1}{\PYZsh{} desloca eixo para x = 0}
\PY{n}{ax}\PY{o}{.}\PY{n}{spines}\PY{p}{[}\PY{l+s+s1}{\PYZsq{}}\PY{l+s+s1}{left}\PY{l+s+s1}{\PYZsq{}}\PY{p}{]}\PY{o}{.}\PY{n}{set\PYZus{}position}\PY{p}{(}\PY{p}{(}\PY{l+s+s1}{\PYZsq{}}\PY{l+s+s1}{data}\PY{l+s+s1}{\PYZsq{}}\PY{p}{,}\PY{l+m+mi}{0}\PY{p}{)}\PY{p}{)} \PY{c+c1}{\PYZsh{} desloca eixo para y = 0}

\PY{n}{ax}\PY{o}{.}\PY{n}{xaxis}\PY{o}{.}\PY{n}{set\PYZus{}ticks\PYZus{}position}\PY{p}{(}\PY{l+s+s1}{\PYZsq{}}\PY{l+s+s1}{top}\PY{l+s+s1}{\PYZsq{}}\PY{p}{)} \PY{c+c1}{\PYZsh{} desloca marcações para cima}
\PY{n}{ax}\PY{o}{.}\PY{n}{yaxis}\PY{o}{.}\PY{n}{set\PYZus{}ticks\PYZus{}position}\PY{p}{(}\PY{l+s+s1}{\PYZsq{}}\PY{l+s+s1}{right}\PY{l+s+s1}{\PYZsq{}}\PY{p}{)} \PY{c+c1}{\PYZsh{} desloca marcações para a direita}

\PY{n}{plt}\PY{o}{.}\PY{n}{xticks}\PY{p}{(}\PY{p}{[}\PY{o}{\PYZhy{}}\PY{l+m+mi}{2}\PY{p}{,}\PY{l+m+mi}{0}\PY{p}{,}\PY{l+m+mi}{2}\PY{p}{]}\PY{p}{)} \PY{c+c1}{\PYZsh{} altera ticks de x}
\PY{n}{ax}\PY{o}{.}\PY{n}{set\PYZus{}xticklabels}\PY{p}{(}\PY{p}{[}\PY{l+s+s1}{\PYZsq{}}\PY{l+s+s1}{esq.}\PY{l+s+s1}{\PYZsq{}}\PY{p}{,}\PY{l+s+s1}{\PYZsq{}}\PY{l+s+s1}{zero}\PY{l+s+s1}{\PYZsq{}}\PY{p}{,}\PY{l+s+s1}{\PYZsq{}}\PY{l+s+s1}{dir.}\PY{l+s+s1}{\PYZsq{}}\PY{p}{]}\PY{p}{)} \PY{c+c1}{\PYZsh{} altera ticklabels de x}

\PY{n}{plt}\PY{o}{.}\PY{n}{yticks}\PY{p}{(}\PY{p}{[}\PY{o}{\PYZhy{}}\PY{l+m+mf}{0.4}\PY{p}{,}\PY{l+m+mi}{0}\PY{p}{,}\PY{l+m+mf}{0.4}\PY{p}{]}\PY{p}{)} \PY{c+c1}{\PYZsh{} altera ticks de y}
\PY{n}{ax}\PY{o}{.}\PY{n}{set\PYZus{}yticklabels}\PY{p}{(}\PY{p}{[}\PY{l+s+s1}{\PYZsq{}}\PY{l+s+s1}{sup.}\PY{l+s+s1}{\PYZsq{}}\PY{p}{,}\PY{l+s+s1}{\PYZsq{}}\PY{l+s+s1}{zero}\PY{l+s+s1}{\PYZsq{}}\PY{p}{,}\PY{l+s+s1}{\PYZsq{}}\PY{l+s+s1}{inf.}\PY{l+s+s1}{\PYZsq{}}\PY{p}{]}\PY{p}{)}\PY{p}{;} \PY{c+c1}{\PYZsh{} altera ticklabels de y}
\end{Verbatim}
\end{tcolorbox}

    \begin{center}
    \adjustimage{max size={0.9\linewidth}{0.9\paperheight}}{04b-plotagem-matplotlib_files/04b-plotagem-matplotlib_35_0.png}
    \end{center}
    { \hspace*{\fill} \\}
    
    \hypertarget{inseruxe7uxe3o-de-legendas}{%
\subsubsection{Inserção de legendas}\label{inseruxe7uxe3o-de-legendas}}

Para criarmos:

\begin{itemize}
\tightlist
\item
  uma legenda para os gráficos, usamos \texttt{legend}.
\item
  uma legenda para o eixo x, usamos \texttt{xlabel}
\item
  uma legenda para o eixo y, usamos \texttt{ylabel}
\item
  um título para o gráfico, usamos \texttt{title}
\end{itemize}

    \textbf{Exemplo:} plote o gráfico da reta \(f_1(x) = x + 1\) e da reta
\(f_2(x) = 1 - x\) e adicione uma legenda com cores azul e laranja.

    \begin{tcolorbox}[breakable, size=fbox, boxrule=1pt, pad at break*=1mm,colback=cellbackground, colframe=cellborder]
\prompt{In}{incolor}{18}{\hspace{4pt}}
\begin{Verbatim}[commandchars=\\\{\}]
\PY{n}{plt}\PY{o}{.}\PY{n}{plot}\PY{p}{(}\PY{n}{x}\PY{p}{,}\PY{n}{x}\PY{o}{+}\PY{l+m+mi}{1}\PY{p}{,}\PY{l+s+s1}{\PYZsq{}}\PY{l+s+s1}{\PYZhy{}b}\PY{l+s+s1}{\PYZsq{}}\PY{p}{,}\PY{n}{label}\PY{o}{=}\PY{l+s+s1}{\PYZsq{}}\PY{l+s+s1}{y = x + 1}\PY{l+s+s1}{\PYZsq{}}\PY{p}{)}
\PY{n}{plt}\PY{o}{.}\PY{n}{plot}\PY{p}{(}\PY{n}{x}\PY{p}{,}\PY{l+m+mi}{1}\PY{o}{\PYZhy{}}\PY{n}{x}\PY{p}{,}\PY{n}{c}\PY{o}{=}\PY{p}{[}\PY{l+m+mf}{1.0}\PY{p}{,}\PY{l+m+mf}{0.5}\PY{p}{,}\PY{l+m+mf}{0.0}\PY{p}{]}\PY{p}{,}\PY{n}{label}\PY{o}{=}\PY{l+s+s1}{\PYZsq{}}\PY{l+s+s1}{y = x + 1}\PY{l+s+s1}{\PYZsq{}}\PY{p}{)}\PY{p}{;} \PY{c+c1}{\PYZsh{} laranja: 100\PYZpc{} de vermelho, 50\PYZpc{} verde}
\PY{n}{plt}\PY{o}{.}\PY{n}{legend}\PY{p}{(}\PY{n}{loc}\PY{o}{=}\PY{l+s+s1}{\PYZsq{}}\PY{l+s+s1}{best}\PY{l+s+s1}{\PYZsq{}}\PY{p}{)} \PY{c+c1}{\PYZsh{} \PYZsq{}loc=best\PYZsq{} : melhor localização da legenda}
\PY{n}{plt}\PY{o}{.}\PY{n}{xlabel}\PY{p}{(}\PY{l+s+s1}{\PYZsq{}}\PY{l+s+s1}{x}\PY{l+s+s1}{\PYZsq{}}\PY{p}{)}\PY{p}{;} \PY{n}{plt}\PY{o}{.}\PY{n}{ylabel}\PY{p}{(}\PY{l+s+s1}{\PYZsq{}}\PY{l+s+s1}{y}\PY{l+s+s1}{\PYZsq{}}\PY{p}{)}\PY{p}{;} \PY{n}{plt}\PY{o}{.}\PY{n}{title}\PY{p}{(}\PY{l+s+s1}{\PYZsq{}}\PY{l+s+s1}{Gráfico de duas retas}\PY{l+s+s1}{\PYZsq{}}\PY{p}{)}\PY{p}{;}
\end{Verbatim}
\end{tcolorbox}

    \begin{center}
    \adjustimage{max size={0.9\linewidth}{0.9\paperheight}}{04b-plotagem-matplotlib_files/04b-plotagem-matplotlib_38_0.png}
    \end{center}
    { \hspace*{\fill} \\}
    
    \hypertarget{localizauxe7uxe3o-de-legendas}{%
\paragraph{Localização de
legendas}\label{localizauxe7uxe3o-de-legendas}}

Use \texttt{loc=valor} para especificar onde posicionar a legenda. Use
\texttt{plt.legend?} para verificar as posições disponíveis para
\texttt{valor}. Vide tabela de valores \texttt{Location\ String} e
\texttt{Location\ Code}.

    \begin{tcolorbox}[breakable, size=fbox, boxrule=1pt, pad at break*=1mm,colback=cellbackground, colframe=cellborder]
\prompt{In}{incolor}{19}{\hspace{4pt}}
\begin{Verbatim}[commandchars=\\\{\}]
\PY{n}{plt}\PY{o}{.}\PY{n}{plot}\PY{p}{(}\PY{n}{np}\PY{o}{.}\PY{n}{nan}\PY{p}{,}\PY{n}{np}\PY{o}{.}\PY{n}{nan}\PY{p}{,}\PY{n}{label}\PY{o}{=}\PY{l+s+s1}{\PYZsq{}}\PY{l+s+s1}{upper right}\PY{l+s+s1}{\PYZsq{}}\PY{p}{)}\PY{p}{;} \PY{c+c1}{\PYZsh{} nan : not a number}
\PY{n}{plt}\PY{o}{.}\PY{n}{legend}\PY{p}{(}\PY{n}{loc}\PY{o}{=}\PY{l+m+mi}{1}\PY{p}{)}\PY{p}{;} \PY{c+c1}{\PYZsh{} usando número}
\end{Verbatim}
\end{tcolorbox}

    \begin{center}
    \adjustimage{max size={0.9\linewidth}{0.9\paperheight}}{04b-plotagem-matplotlib_files/04b-plotagem-matplotlib_40_0.png}
    \end{center}
    { \hspace*{\fill} \\}
    
    \begin{tcolorbox}[breakable, size=fbox, boxrule=1pt, pad at break*=1mm,colback=cellbackground, colframe=cellborder]
\prompt{In}{incolor}{20}{\hspace{4pt}}
\begin{Verbatim}[commandchars=\\\{\}]
\PY{n}{plt}\PY{o}{.}\PY{n}{plot}\PY{p}{(}\PY{n}{np}\PY{o}{.}\PY{n}{nan}\PY{p}{,}\PY{n}{np}\PY{o}{.}\PY{n}{nan}\PY{p}{,}\PY{n}{label}\PY{o}{=}\PY{l+s+s1}{\PYZsq{}}\PY{l+s+s1}{loc=1}\PY{l+s+s1}{\PYZsq{}}\PY{p}{)}\PY{p}{;}
\PY{n}{plt}\PY{o}{.}\PY{n}{legend}\PY{p}{(}\PY{n}{loc}\PY{o}{=}\PY{l+s+s1}{\PYZsq{}}\PY{l+s+s1}{upper right}\PY{l+s+s1}{\PYZsq{}}\PY{p}{)}\PY{p}{;} \PY{c+c1}{\PYZsh{} usando a string correspondente}
\end{Verbatim}
\end{tcolorbox}

    \begin{center}
    \adjustimage{max size={0.9\linewidth}{0.9\paperheight}}{04b-plotagem-matplotlib_files/04b-plotagem-matplotlib_41_0.png}
    \end{center}
    { \hspace*{\fill} \\}
    
    \hypertarget{alterauxe7uxe3o-de-tamanho-de-fonte}{%
\subsubsection{Alteração de tamanho de
fonte}\label{alterauxe7uxe3o-de-tamanho-de-fonte}}

Para alterar o tamanho da fonte de legendas, use \texttt{fontsize}.

    \begin{tcolorbox}[breakable, size=fbox, boxrule=1pt, pad at break*=1mm,colback=cellbackground, colframe=cellborder]
\prompt{In}{incolor}{21}{\hspace{4pt}}
\begin{Verbatim}[commandchars=\\\{\}]
\PY{n}{plt}\PY{o}{.}\PY{n}{plot}\PY{p}{(}\PY{n}{np}\PY{o}{.}\PY{n}{nan}\PY{p}{,}\PY{n}{np}\PY{o}{.}\PY{n}{nan}\PY{p}{,}\PY{n}{label}\PY{o}{=}\PY{l+s+s1}{\PYZsq{}}\PY{l+s+s1}{legenda}\PY{l+s+s1}{\PYZsq{}}\PY{p}{)}\PY{p}{;}

\PY{n}{FSx}\PY{p}{,} \PY{n}{FSy}\PY{p}{,} \PY{n}{FSleg}\PY{p}{,} \PY{n}{FStit} \PY{o}{=} \PY{l+m+mi}{10}\PY{p}{,} \PY{l+m+mi}{20}\PY{p}{,} \PY{l+m+mi}{30}\PY{p}{,} \PY{l+m+mi}{40}

\PY{n}{plt}\PY{o}{.}\PY{n}{xlabel}\PY{p}{(}\PY{l+s+s1}{\PYZsq{}}\PY{l+s+s1}{Eixo x}\PY{l+s+s1}{\PYZsq{}}\PY{p}{,}\PY{n}{c}\PY{o}{=}\PY{l+s+s1}{\PYZsq{}}\PY{l+s+s1}{b}\PY{l+s+s1}{\PYZsq{}}\PY{p}{,} \PY{n}{fontsize}\PY{o}{=}\PY{n}{FSx}\PY{p}{)}
\PY{n}{plt}\PY{o}{.}\PY{n}{ylabel}\PY{p}{(}\PY{l+s+s1}{\PYZsq{}}\PY{l+s+s1}{Eixo y}\PY{l+s+s1}{\PYZsq{}}\PY{p}{,}\PY{n}{c}\PY{o}{=}\PY{l+s+s1}{\PYZsq{}}\PY{l+s+s1}{g}\PY{l+s+s1}{\PYZsq{}}\PY{p}{,} \PY{n}{fontsize}\PY{o}{=}\PY{n}{FSy}\PY{p}{)}
\PY{n}{plt}\PY{o}{.}\PY{n}{legend}\PY{p}{(}\PY{n}{loc}\PY{o}{=}\PY{l+s+s1}{\PYZsq{}}\PY{l+s+s1}{center}\PY{l+s+s1}{\PYZsq{}}\PY{p}{,} \PY{n}{fontsize}\PY{o}{=}\PY{n}{FSleg}\PY{p}{)}\PY{p}{;} 
\PY{n}{plt}\PY{o}{.}\PY{n}{title}\PY{p}{(}\PY{l+s+s1}{\PYZsq{}}\PY{l+s+s1}{Título}\PY{l+s+s1}{\PYZsq{}}\PY{p}{,} \PY{n}{c}\PY{o}{=}\PY{l+s+s1}{\PYZsq{}}\PY{l+s+s1}{c}\PY{l+s+s1}{\PYZsq{}}\PY{p}{,} \PY{n}{fontsize}\PY{o}{=}\PY{n}{FStit}\PY{p}{)}\PY{p}{;}
\end{Verbatim}
\end{tcolorbox}

    \begin{center}
    \adjustimage{max size={0.9\linewidth}{0.9\paperheight}}{04b-plotagem-matplotlib_files/04b-plotagem-matplotlib_43_0.png}
    \end{center}
    { \hspace*{\fill} \\}
    
    \hypertarget{anotauxe7uxf5es-simples}{%
\subsubsection{Anotações simples}\label{anotauxe7uxf5es-simples}}

Podemos incluir anotações em gráficos com a função
\texttt{annotate(texto,xref,yref)}

    \begin{tcolorbox}[breakable, size=fbox, boxrule=1pt, pad at break*=1mm,colback=cellbackground, colframe=cellborder]
\prompt{In}{incolor}{22}{\hspace{4pt}}
\begin{Verbatim}[commandchars=\\\{\}]
\PY{n}{plt}\PY{o}{.}\PY{n}{plot}\PY{p}{(}\PY{n}{np}\PY{o}{.}\PY{n}{nan}\PY{p}{,}\PY{n}{np}\PY{o}{.}\PY{n}{nan}\PY{p}{)}\PY{p}{;}
\PY{n}{plt}\PY{o}{.}\PY{n}{annotate}\PY{p}{(}\PY{l+s+s1}{\PYZsq{}}\PY{l+s+s1}{P (0.5,0.5)}\PY{l+s+s1}{\PYZsq{}}\PY{p}{,}\PY{p}{(}\PY{l+m+mf}{0.5}\PY{p}{,}\PY{l+m+mf}{0.5}\PY{p}{)}\PY{p}{)}\PY{p}{;}
\PY{n}{plt}\PY{o}{.}\PY{n}{annotate}\PY{p}{(}\PY{l+s+s1}{\PYZsq{}}\PY{l+s+s1}{Q (0.1,0.8)}\PY{l+s+s1}{\PYZsq{}}\PY{p}{,}\PY{p}{(}\PY{l+m+mf}{0.1}\PY{p}{,}\PY{l+m+mf}{0.8}\PY{p}{)}\PY{p}{)}\PY{p}{;}
\end{Verbatim}
\end{tcolorbox}

    \begin{center}
    \adjustimage{max size={0.9\linewidth}{0.9\paperheight}}{04b-plotagem-matplotlib_files/04b-plotagem-matplotlib_45_0.png}
    \end{center}
    { \hspace*{\fill} \\}
    
    \textbf{Exemplo}: gere um conjunto de 10 pontos \((x,y)\) aleatórios em
que \(0.2 < x,y < 0.8\) e anote-os no plano.

    \begin{tcolorbox}[breakable, size=fbox, boxrule=1pt, pad at break*=1mm,colback=cellbackground, colframe=cellborder]
\prompt{In}{incolor}{23}{\hspace{4pt}}
\begin{Verbatim}[commandchars=\\\{\}]
\PY{c+c1}{\PYZsh{} gera uma lista de 10 pontos satisfazendo a condição}
\PY{n}{P} \PY{o}{=} \PY{p}{[}\PY{p}{]}
\PY{k}{while} \PY{n+nb}{len}\PY{p}{(}\PY{n}{P}\PY{p}{)} \PY{o}{!=} \PY{l+m+mi}{10}\PY{p}{:}
    \PY{n}{xy} \PY{o}{=} \PY{n}{np}\PY{o}{.}\PY{n}{round}\PY{p}{(}\PY{n}{np}\PY{o}{.}\PY{n}{random}\PY{o}{.}\PY{n}{rand}\PY{p}{(}\PY{l+m+mi}{2}\PY{p}{)}\PY{p}{,}\PY{l+m+mi}{1}\PY{p}{)} 
    \PY{n}{test} \PY{o}{=} \PY{n}{np}\PY{o}{.}\PY{n}{all}\PY{p}{(} \PY{p}{(}\PY{n}{xy} \PY{o}{\PYZgt{}} \PY{l+m+mf}{0.2}\PY{p}{)} \PY{o}{\PYZam{}} \PY{p}{(}\PY{n}{xy} \PY{o}{\PYZlt{}} \PY{l+m+mf}{0.8}\PY{p}{)} \PY{p}{)}
    \PY{k}{if} \PY{n}{test}\PY{p}{:}
        \PY{n}{P}\PY{o}{.}\PY{n}{append}\PY{p}{(}\PY{n+nb}{tuple}\PY{p}{(}\PY{n}{xy}\PY{p}{)}\PY{p}{)}

\PY{c+c1}{\PYZsh{} plota o plano}
\PY{n}{plt}\PY{o}{.}\PY{n}{figure}\PY{p}{(}\PY{n}{figsize}\PY{o}{=}\PY{p}{(}\PY{l+m+mi}{8}\PY{p}{,}\PY{l+m+mi}{8}\PY{p}{)}\PY{p}{)}
\PY{n}{plt}\PY{o}{.}\PY{n}{xlim}\PY{p}{(}\PY{l+m+mi}{0}\PY{p}{,}\PY{l+m+mi}{1}\PY{p}{)}
\PY{n}{plt}\PY{o}{.}\PY{n}{ylim}\PY{p}{(}\PY{l+m+mi}{0}\PY{p}{,}\PY{l+m+mi}{1}\PY{p}{)}

\PY{k}{for} \PY{n}{ponto} \PY{o+ow}{in} \PY{n}{P}\PY{p}{:}    
    \PY{n}{plt}\PY{o}{.}\PY{n}{plot}\PY{p}{(}\PY{n}{ponto}\PY{p}{[}\PY{l+m+mi}{0}\PY{p}{]}\PY{p}{,}\PY{n}{ponto}\PY{p}{[}\PY{l+m+mi}{1}\PY{p}{]}\PY{p}{,}\PY{l+s+s1}{\PYZsq{}}\PY{l+s+s1}{o}\PY{l+s+s1}{\PYZsq{}}\PY{p}{)}
    \PY{n}{plt}\PY{o}{.}\PY{n}{annotate}\PY{p}{(}\PY{n}{f}\PY{l+s+s1}{\PYZsq{}}\PY{l+s+s1}{(}\PY{l+s+si}{\PYZob{}ponto[0]\PYZcb{}}\PY{l+s+s1}{,}\PY{l+s+si}{\PYZob{}ponto[1]\PYZcb{}}\PY{l+s+s1}{)}\PY{l+s+s1}{\PYZsq{}}\PY{p}{,}\PY{n}{ponto}\PY{p}{,}\PY{n}{fontsize}\PY{o}{=}\PY{l+m+mi}{14}\PY{p}{)}
\end{Verbatim}
\end{tcolorbox}

    \begin{center}
    \adjustimage{max size={0.9\linewidth}{0.9\paperheight}}{04b-plotagem-matplotlib_files/04b-plotagem-matplotlib_47_0.png}
    \end{center}
    { \hspace*{\fill} \\}
    
    \textbf{Problema:} o código acima tem um problema. Verifique que
\texttt{len(P)\ =\ 10}, mas ele não plota os 10 pontos como gostaríamos
de ver. Descubra o que está acontecendo e proponha uma solução.

    \hypertarget{multiplotagem-e-eixos}{%
\subsection{Multiplotagem e eixos}\label{multiplotagem-e-eixos}}

No matplotlib, podemos trabalhar com a função \texttt{subplot(m,n,p)}
para criar múltiplas figuras e eixos independentes como se cada figura
fosse um elemento de uma grande ``matriz de figuras'' de \texttt{m}
linhas e \texttt{n} colunas, enquanto \texttt{p} é o índice da figura
(este valor será no máximo o produto \texttt{mxn}). A função funciona da
seguinte forma.

\begin{itemize}
\item
  Exemplo 1: suponha que você queira criar 3 figuras e dispô-las em uma
  única linha. Neste caso, \texttt{m\ =\ 1}, \texttt{n\ =\ 3} e
  \texttt{p} variará de 1 a 3, visto que \texttt{mxn\ =\ 3}.
\item
  Exemplo 2: suponha que você queira criar 6 figuras e dispô-las em 2
  linhas e 3 colunas. Neste caso, \texttt{m\ =\ 2}, \texttt{n\ =\ 3} e
  \texttt{p} variará de 1 a 6, visto que \texttt{mxn\ =\ 6}.
\item
  Exemplo 3: suponha que você queira criar 12 figuras e dispô-las em 4
  linhas e 3 colunas. Neste caso, \texttt{m\ =\ 4}, \texttt{n\ =\ 3} e
  \texttt{p} variará de 1 a 12, visto que \texttt{mxn\ =\ 12}.
\end{itemize}

Cada plotagem possui seu eixo independentemente da outra.

    \textbf{Exemplo 1:} gráfico de 1 reta, 1 parábola e 1 polinômio cúbico
lado a lado.

    \begin{tcolorbox}[breakable, size=fbox, boxrule=1pt, pad at break*=1mm,colback=cellbackground, colframe=cellborder]
\prompt{In}{incolor}{24}{\hspace{4pt}}
\begin{Verbatim}[commandchars=\\\{\}]
\PY{n}{x} \PY{o}{=} \PY{n}{np}\PY{o}{.}\PY{n}{linspace}\PY{p}{(}\PY{o}{\PYZhy{}}\PY{l+m+mi}{5}\PY{p}{,}\PY{l+m+mi}{5}\PY{p}{,}\PY{l+m+mi}{20}\PY{p}{)}

\PY{n}{plt}\PY{o}{.}\PY{n}{figure}\PY{p}{(}\PY{n}{figsize}\PY{o}{=}\PY{p}{(}\PY{l+m+mi}{15}\PY{p}{,}\PY{l+m+mi}{4}\PY{p}{)}\PY{p}{)}

\PY{c+c1}{\PYZsh{} aqui p = 1}
\PY{n}{plt}\PY{o}{.}\PY{n}{subplot}\PY{p}{(}\PY{l+m+mi}{1}\PY{p}{,}\PY{l+m+mi}{3}\PY{p}{,}\PY{l+m+mi}{1}\PY{p}{)} \PY{c+c1}{\PYZsh{} plt.subplot(131) também é válida}
\PY{n}{plt}\PY{o}{.}\PY{n}{plot}\PY{p}{(}\PY{n}{x}\PY{p}{,}\PY{l+m+mi}{2}\PY{o}{*}\PY{n}{x}\PY{o}{\PYZhy{}}\PY{l+m+mi}{1}\PY{p}{,}\PY{n}{c}\PY{o}{=}\PY{l+s+s1}{\PYZsq{}}\PY{l+s+s1}{r}\PY{l+s+s1}{\PYZsq{}}\PY{p}{,}\PY{n}{marker}\PY{o}{=}\PY{l+s+s1}{\PYZsq{}}\PY{l+s+s1}{\PYZca{}}\PY{l+s+s1}{\PYZsq{}}\PY{p}{)}
\PY{n}{plt}\PY{o}{.}\PY{n}{title}\PY{p}{(}\PY{l+s+s1}{\PYZsq{}}\PY{l+s+s1}{\PYZdl{}y=2x\PYZhy{}1\PYZdl{}}\PY{l+s+s1}{\PYZsq{}}\PY{p}{)}

\PY{c+c1}{\PYZsh{} aqui p = 2}
\PY{n}{plt}\PY{o}{.}\PY{n}{subplot}\PY{p}{(}\PY{l+m+mi}{1}\PY{p}{,}\PY{l+m+mi}{3}\PY{p}{,}\PY{l+m+mi}{2}\PY{p}{)} \PY{c+c1}{\PYZsh{} plt.subplot(131) também é válida}
\PY{n}{plt}\PY{o}{.}\PY{n}{plot}\PY{p}{(}\PY{n}{x}\PY{p}{,}\PY{l+m+mi}{3}\PY{o}{*}\PY{n}{x}\PY{o}{*}\PY{o}{*}\PY{l+m+mi}{2} \PY{o}{\PYZhy{}} \PY{l+m+mi}{2}\PY{o}{*}\PY{n}{x} \PY{o}{\PYZhy{}} \PY{l+m+mi}{1}\PY{p}{,}\PY{n}{c}\PY{o}{=}\PY{l+s+s1}{\PYZsq{}}\PY{l+s+s1}{g}\PY{l+s+s1}{\PYZsq{}}\PY{p}{,}\PY{n}{marker}\PY{o}{=}\PY{l+s+s1}{\PYZsq{}}\PY{l+s+s1}{o}\PY{l+s+s1}{\PYZsq{}}\PY{p}{)}
\PY{n}{plt}\PY{o}{.}\PY{n}{title}\PY{p}{(}\PY{l+s+s1}{\PYZsq{}}\PY{l+s+s1}{\PYZdl{}y=3x\PYZca{}2 \PYZhy{} 2x \PYZhy{} 1\PYZdl{}}\PY{l+s+s1}{\PYZsq{}}\PY{p}{)}

\PY{c+c1}{\PYZsh{} aqui p = 3}
\PY{n}{plt}\PY{o}{.}\PY{n}{subplot}\PY{p}{(}\PY{l+m+mi}{1}\PY{p}{,}\PY{l+m+mi}{3}\PY{p}{,}\PY{l+m+mi}{3}\PY{p}{)} \PY{c+c1}{\PYZsh{} plt.subplot(131) também é válida}
\PY{n}{plt}\PY{o}{.}\PY{n}{plot}\PY{p}{(}\PY{n}{x}\PY{p}{,}\PY{l+m+mi}{1}\PY{o}{/}\PY{l+m+mi}{2}\PY{o}{*}\PY{n}{x}\PY{o}{*}\PY{o}{*}\PY{l+m+mi}{3} \PY{o}{+} \PY{l+m+mi}{3}\PY{o}{*}\PY{n}{x}\PY{o}{*}\PY{o}{*}\PY{l+m+mi}{2} \PY{o}{\PYZhy{}} \PY{l+m+mi}{2}\PY{o}{*}\PY{n}{x} \PY{o}{\PYZhy{}} \PY{l+m+mi}{1}\PY{p}{,}\PY{n}{c}\PY{o}{=}\PY{l+s+s1}{\PYZsq{}}\PY{l+s+s1}{b}\PY{l+s+s1}{\PYZsq{}}\PY{p}{,}\PY{n}{marker}\PY{o}{=}\PY{l+s+s1}{\PYZsq{}}\PY{l+s+s1}{*}\PY{l+s+s1}{\PYZsq{}}\PY{p}{)}
\PY{n}{plt}\PY{o}{.}\PY{n}{title}\PY{p}{(}\PY{l+s+s1}{\PYZsq{}}\PY{l+s+s1}{\PYZdl{}y=1/2x\PYZca{}3 + 3x\PYZca{}2 \PYZhy{} 2x \PYZhy{} 1\PYZdl{}}\PY{l+s+s1}{\PYZsq{}}\PY{p}{)}\PY{p}{;}
\end{Verbatim}
\end{tcolorbox}

    \begin{center}
    \adjustimage{max size={0.9\linewidth}{0.9\paperheight}}{04b-plotagem-matplotlib_files/04b-plotagem-matplotlib_51_0.png}
    \end{center}
    { \hspace*{\fill} \\}
    
    \textbf{Exemplo 2:} gráficos de \{\(sen(x)\), \(sen(2x)\), \(sen(3x)\)\}
e \{\(cos(x)\), \(cos(2x)\), \(cos(3x)\)\} dispostos em matriz 2x3.

    \begin{tcolorbox}[breakable, size=fbox, boxrule=1pt, pad at break*=1mm,colback=cellbackground, colframe=cellborder]
\prompt{In}{incolor}{25}{\hspace{4pt}}
\begin{Verbatim}[commandchars=\\\{\}]
\PY{n}{plt}\PY{o}{.}\PY{n}{figure}\PY{p}{(}\PY{n}{figsize}\PY{o}{=}\PY{p}{(}\PY{l+m+mi}{15}\PY{p}{,}\PY{l+m+mi}{4}\PY{p}{)}\PY{p}{)}
\PY{n}{plt}\PY{o}{.}\PY{n}{subplots\PYZus{}adjust}\PY{p}{(}\PY{n}{top}\PY{o}{=}\PY{l+m+mf}{2.5}\PY{p}{,}\PY{n}{right}\PY{o}{=}\PY{l+m+mf}{1.2}\PY{p}{)} \PY{c+c1}{\PYZsh{} ajusta a separação dos plots individuais}

\PY{k}{def} \PY{n+nf}{sencosx}\PY{p}{(}\PY{n}{p}\PY{p}{)}\PY{p}{:}
    \PY{n}{x} \PY{o}{=} \PY{n}{np}\PY{o}{.}\PY{n}{linspace}\PY{p}{(}\PY{l+m+mi}{0}\PY{p}{,}\PY{l+m+mi}{2}\PY{o}{*}\PY{n}{np}\PY{o}{.}\PY{n}{pi}\PY{p}{,}\PY{l+m+mi}{50}\PY{p}{)}    
    \PY{n}{plt}\PY{o}{.}\PY{n}{subplot}\PY{p}{(}\PY{l+m+mi}{2}\PY{p}{,}\PY{l+m+mi}{3}\PY{p}{,}\PY{n}{p}\PY{p}{)}
    \PY{k}{if} \PY{n}{p} \PY{o}{\PYZlt{}}\PY{o}{=} \PY{l+m+mi}{3}\PY{p}{:}            
        \PY{n}{plt}\PY{o}{.}\PY{n}{plot}\PY{p}{(}\PY{n}{x}\PY{p}{,}\PY{n}{np}\PY{o}{.}\PY{n}{sin}\PY{p}{(}\PY{n}{p}\PY{o}{*}\PY{n}{x}\PY{p}{)}\PY{p}{,}\PY{n}{c}\PY{o}{=}\PY{p}{[}\PY{n}{p}\PY{o}{/}\PY{l+m+mi}{4}\PY{p}{,}\PY{n}{p}\PY{o}{/}\PY{l+m+mi}{5}\PY{p}{,}\PY{n}{p}\PY{o}{/}\PY{l+m+mi}{6}\PY{p}{]}\PY{p}{,}\PY{n}{label}\PY{o}{=}\PY{n}{f}\PY{l+s+s1}{\PYZsq{}}\PY{l+s+s1}{\PYZdl{}sen(}\PY{l+s+si}{\PYZob{}p\PYZcb{}}\PY{l+s+s1}{x)\PYZdl{}}\PY{l+s+s1}{\PYZsq{}}\PY{p}{)}
        \PY{n}{plt}\PY{o}{.}\PY{n}{title}\PY{p}{(}\PY{n}{f}\PY{l+s+s1}{\PYZsq{}}\PY{l+s+s1}{subplot(2,3,}\PY{l+s+si}{\PYZob{}p\PYZcb{}}\PY{l+s+s1}{)}\PY{l+s+s1}{\PYZsq{}}\PY{p}{)}\PY{p}{;}
    \PY{k}{else}\PY{p}{:}        
        \PY{n}{plt}\PY{o}{.}\PY{n}{title}\PY{p}{(}\PY{n}{f}\PY{l+s+s1}{\PYZsq{}}\PY{l+s+s1}{subplot(2,3,}\PY{l+s+si}{\PYZob{}p\PYZcb{}}\PY{l+s+s1}{)}\PY{l+s+s1}{\PYZsq{}}\PY{p}{)}\PY{p}{;}
        \PY{n}{p}\PY{o}{\PYZhy{}}\PY{o}{=}\PY{l+m+mi}{3} \PY{c+c1}{\PYZsh{} }
        \PY{n}{plt}\PY{o}{.}\PY{n}{plot}\PY{p}{(}\PY{n}{x}\PY{p}{,}\PY{n}{np}\PY{o}{.}\PY{n}{cos}\PY{p}{(}\PY{n}{p}\PY{o}{*}\PY{n}{x}\PY{p}{)}\PY{p}{,}\PY{n}{c}\PY{o}{=}\PY{p}{[}\PY{n}{p}\PY{o}{/}\PY{l+m+mi}{9}\PY{p}{,}\PY{n}{p}\PY{o}{/}\PY{l+m+mi}{7}\PY{p}{,}\PY{n}{p}\PY{o}{/}\PY{l+m+mi}{8}\PY{p}{]}\PY{p}{,}\PY{n}{label}\PY{o}{=}\PY{n}{f}\PY{l+s+s1}{\PYZsq{}}\PY{l+s+s1}{\PYZdl{}cos(}\PY{l+s+si}{\PYZob{}p\PYZcb{}}\PY{l+s+s1}{x)\PYZdl{}}\PY{l+s+s1}{\PYZsq{}}\PY{p}{)}
    
    \PY{n}{plt}\PY{o}{.}\PY{n}{legend}\PY{p}{(}\PY{n}{loc}\PY{o}{=}\PY{l+m+mi}{0}\PY{p}{,}\PY{n}{fontsize}\PY{o}{=}\PY{l+m+mi}{8}\PY{p}{)}
    \PY{n}{plt}\PY{o}{.}\PY{n}{xlabel}\PY{p}{(}\PY{l+s+s1}{\PYZsq{}}\PY{l+s+s1}{x}\PY{l+s+s1}{\PYZsq{}}\PY{p}{)}\PY{p}{;} \PY{n}{plt}\PY{o}{.}\PY{n}{ylabel}\PY{p}{(}\PY{l+s+s1}{\PYZsq{}}\PY{l+s+s1}{y}\PY{l+s+s1}{\PYZsq{}}\PY{p}{)}\PY{p}{;} 
        
\PY{c+c1}{\PYZsh{} plotagem     }
\PY{k}{for} \PY{n}{p} \PY{o+ow}{in} \PY{n+nb}{range}\PY{p}{(}\PY{l+m+mi}{1}\PY{p}{,}\PY{l+m+mi}{7}\PY{p}{)}\PY{p}{:}
    \PY{n}{sencosx}\PY{p}{(}\PY{n}{p}\PY{p}{)}
\end{Verbatim}
\end{tcolorbox}

    \begin{center}
    \adjustimage{max size={0.9\linewidth}{0.9\paperheight}}{04b-plotagem-matplotlib_files/04b-plotagem-matplotlib_53_0.png}
    \end{center}
    { \hspace*{\fill} \\}
    
    \textbf{Exemplo 3:} gráficos de um ponto isolado em matriz 4 x 3.

    \begin{tcolorbox}[breakable, size=fbox, boxrule=1pt, pad at break*=1mm,colback=cellbackground, colframe=cellborder]
\prompt{In}{incolor}{26}{\hspace{4pt}}
\begin{Verbatim}[commandchars=\\\{\}]
\PY{n}{plt}\PY{o}{.}\PY{n}{figure}\PY{p}{(}\PY{n}{figsize}\PY{o}{=}\PY{p}{(}\PY{l+m+mi}{15}\PY{p}{,}\PY{l+m+mi}{4}\PY{p}{)}\PY{p}{)}
\PY{n}{m}\PY{p}{,}\PY{n}{n} \PY{o}{=} \PY{l+m+mi}{4}\PY{p}{,}\PY{l+m+mi}{3}
\PY{k}{def} \PY{n+nf}{star}\PY{p}{(}\PY{n}{p}\PY{p}{)}\PY{p}{:}
    \PY{n}{plt}\PY{o}{.}\PY{n}{subplot}\PY{p}{(}\PY{n}{m}\PY{p}{,}\PY{n}{n}\PY{p}{,}\PY{n}{p}\PY{p}{)}
    \PY{n}{plt}\PY{o}{.}\PY{n}{axis}\PY{p}{(}\PY{l+s+s1}{\PYZsq{}}\PY{l+s+s1}{off}\PY{l+s+s1}{\PYZsq{}}\PY{p}{)} \PY{c+c1}{\PYZsh{} desliga eixos}
    \PY{n}{plt}\PY{o}{.}\PY{n}{plot}\PY{p}{(}\PY{l+m+mf}{0.5}\PY{p}{,}\PY{l+m+mf}{0.5}\PY{p}{,}\PY{n}{marker}\PY{o}{=}\PY{l+s+s1}{\PYZsq{}}\PY{l+s+s1}{*}\PY{l+s+s1}{\PYZsq{}}\PY{p}{,}\PY{n}{c}\PY{o}{=}\PY{n+nb}{list}\PY{p}{(}\PY{n}{np}\PY{o}{.}\PY{n}{random}\PY{o}{.}\PY{n}{rand}\PY{p}{(}\PY{l+m+mi}{3}\PY{p}{)}\PY{p}{)}\PY{p}{,}\PY{n}{ms}\PY{o}{=}\PY{n}{p}\PY{o}{*}\PY{l+m+mi}{2}\PY{p}{)}
    \PY{n}{plt}\PY{o}{.}\PY{n}{annotate}\PY{p}{(}\PY{n}{f}\PY{l+s+s1}{\PYZsq{}}\PY{l+s+s1}{subplot(}\PY{l+s+si}{\PYZob{}m\PYZcb{}}\PY{l+s+s1}{,}\PY{l+s+si}{\PYZob{}n\PYZcb{}}\PY{l+s+s1}{,}\PY{l+s+si}{\PYZob{}p\PYZcb{}}\PY{l+s+s1}{)}\PY{l+s+s1}{\PYZsq{}}\PY{p}{,}\PY{p}{(}\PY{l+m+mf}{0.5}\PY{p}{,}\PY{l+m+mf}{0.5}\PY{p}{)}\PY{p}{,}\PY{n}{c}\PY{o}{=}\PY{l+s+s1}{\PYZsq{}}\PY{l+s+s1}{g}\PY{l+s+s1}{\PYZsq{}}\PY{p}{,}\PY{n}{fontsize}\PY{o}{=}\PY{l+m+mi}{10}\PY{p}{)}
    
\PY{k}{for} \PY{n}{p} \PY{o+ow}{in} \PY{n+nb}{range}\PY{p}{(}\PY{l+m+mi}{1}\PY{p}{,}\PY{n}{m}\PY{o}{*}\PY{n}{n}\PY{o}{+}\PY{l+m+mi}{1}\PY{p}{)}\PY{p}{:}
    \PY{n}{star}\PY{p}{(}\PY{n}{p}\PY{p}{)}\PY{p}{;}
\end{Verbatim}
\end{tcolorbox}

    \begin{center}
    \adjustimage{max size={0.9\linewidth}{0.9\paperheight}}{04b-plotagem-matplotlib_files/04b-plotagem-matplotlib_55_0.png}
    \end{center}
    { \hspace*{\fill} \\}
    
    \hypertarget{plots-com-gradeado}{%
\subsection{Plots com gradeado}\label{plots-com-gradeado}}

Podemos habilitar o gradeado usando \texttt{grid(b,which,axis)}.

Para especificar o gradeado:

\begin{itemize}
\tightlist
\item
  em ambos os eixos, use
  \texttt{b=\textquotesingle{}True\textquotesingle{}} ou
  \texttt{b=\textquotesingle{}False\textquotesingle{}}.
\item
  maior, menor ou ambos, use
  \texttt{which=\textquotesingle{}major\textquotesingle{}},
  \texttt{which=\textquotesingle{}minor\textquotesingle{}} ou
  \texttt{which=\textquotesingle{}both\textquotesingle{}}.
\item
  nos eixos x, y ou ambos, use
  \texttt{axis=\textquotesingle{}x\textquotesingle{}},
  \texttt{axis=\textquotesingle{}y\textquotesingle{}} ou
  \texttt{axis=\textquotesingle{}both\textquotesingle{}}.
\end{itemize}

    \begin{tcolorbox}[breakable, size=fbox, boxrule=1pt, pad at break*=1mm,colback=cellbackground, colframe=cellborder]
\prompt{In}{incolor}{27}{\hspace{4pt}}
\begin{Verbatim}[commandchars=\\\{\}]
\PY{n}{x} \PY{o}{=} \PY{n}{np}\PY{o}{.}\PY{n}{linspace}\PY{p}{(}\PY{o}{\PYZhy{}}\PY{l+m+mi}{10}\PY{p}{,}\PY{l+m+mi}{10}\PY{p}{)}
\PY{n}{plt}\PY{o}{.}\PY{n}{plot}\PY{p}{(}\PY{n}{x}\PY{p}{,}\PY{n}{x}\PY{p}{)}
\PY{n}{plt}\PY{o}{.}\PY{n}{grid}\PY{p}{(}\PY{k+kc}{True}\PY{p}{)}
\end{Verbatim}
\end{tcolorbox}

    \begin{center}
    \adjustimage{max size={0.9\linewidth}{0.9\paperheight}}{04b-plotagem-matplotlib_files/04b-plotagem-matplotlib_57_0.png}
    \end{center}
    { \hspace*{\fill} \\}
    
    \begin{tcolorbox}[breakable, size=fbox, boxrule=1pt, pad at break*=1mm,colback=cellbackground, colframe=cellborder]
\prompt{In}{incolor}{28}{\hspace{4pt}}
\begin{Verbatim}[commandchars=\\\{\}]
\PY{n}{plt}\PY{o}{.}\PY{n}{plot}\PY{p}{(}\PY{n}{x}\PY{p}{,}\PY{n}{x}\PY{p}{)}
\PY{n}{plt}\PY{o}{.}\PY{n}{grid}\PY{p}{(}\PY{k+kc}{True}\PY{p}{,}\PY{n}{which}\PY{o}{=}\PY{l+s+s1}{\PYZsq{}}\PY{l+s+s1}{major}\PY{l+s+s1}{\PYZsq{}}\PY{p}{,}\PY{n}{axis}\PY{o}{=}\PY{l+s+s1}{\PYZsq{}}\PY{l+s+s1}{x}\PY{l+s+s1}{\PYZsq{}}\PY{p}{)}
\end{Verbatim}
\end{tcolorbox}

    \begin{center}
    \adjustimage{max size={0.9\linewidth}{0.9\paperheight}}{04b-plotagem-matplotlib_files/04b-plotagem-matplotlib_58_0.png}
    \end{center}
    { \hspace*{\fill} \\}
    
    \begin{tcolorbox}[breakable, size=fbox, boxrule=1pt, pad at break*=1mm,colback=cellbackground, colframe=cellborder]
\prompt{In}{incolor}{29}{\hspace{4pt}}
\begin{Verbatim}[commandchars=\\\{\}]
\PY{n}{plt}\PY{o}{.}\PY{n}{plot}\PY{p}{(}\PY{n}{x}\PY{p}{,}\PY{n}{x}\PY{p}{)}
\PY{n}{plt}\PY{o}{.}\PY{n}{grid}\PY{p}{(}\PY{k+kc}{True}\PY{p}{,}\PY{n}{which}\PY{o}{=}\PY{l+s+s1}{\PYZsq{}}\PY{l+s+s1}{major}\PY{l+s+s1}{\PYZsq{}}\PY{p}{,}\PY{n}{axis}\PY{o}{=}\PY{l+s+s1}{\PYZsq{}}\PY{l+s+s1}{y}\PY{l+s+s1}{\PYZsq{}}\PY{p}{)}
\end{Verbatim}
\end{tcolorbox}

    \begin{center}
    \adjustimage{max size={0.9\linewidth}{0.9\paperheight}}{04b-plotagem-matplotlib_files/04b-plotagem-matplotlib_59_0.png}
    \end{center}
    { \hspace*{\fill} \\}
    
    \textbf{Exemplo:} plotagem de gradeado.

Neste exemplo, um eixo abstrato é adicionado sobre a figura (criada
diretamente) origem no ponto (0.025,0.025), largura 0.95 e altura 0.95.

    \begin{tcolorbox}[breakable, size=fbox, boxrule=1pt, pad at break*=1mm,colback=cellbackground, colframe=cellborder]
\prompt{In}{incolor}{30}{\hspace{4pt}}
\begin{Verbatim}[commandchars=\\\{\}]
\PY{n}{ax} \PY{o}{=} \PY{n}{plt}\PY{o}{.}\PY{n}{axes}\PY{p}{(}\PY{p}{[}\PY{l+m+mf}{0.025}\PY{p}{,} \PY{l+m+mf}{0.025}\PY{p}{,} \PY{l+m+mf}{0.95}\PY{p}{,} \PY{l+m+mf}{0.95}\PY{p}{]}\PY{p}{)}
\PY{n}{ax}\PY{o}{.}\PY{n}{set\PYZus{}xlim}\PY{p}{(}\PY{l+m+mi}{0}\PY{p}{,}\PY{l+m+mi}{4}\PY{p}{)}
\PY{n}{ax}\PY{o}{.}\PY{n}{set\PYZus{}ylim}\PY{p}{(}\PY{l+m+mi}{0}\PY{p}{,}\PY{l+m+mi}{3}\PY{p}{)}

\PY{c+c1}{\PYZsh{} MultipleLocator estabelece pontos de referência para divisão da grade}
\PY{n}{ax}\PY{o}{.}\PY{n}{xaxis}\PY{o}{.}\PY{n}{set\PYZus{}major\PYZus{}locator}\PY{p}{(}\PY{n}{plt}\PY{o}{.}\PY{n}{MultipleLocator}\PY{p}{(}\PY{l+m+mf}{1.0}\PY{p}{)}\PY{p}{)} \PY{c+c1}{\PYZsh{} divisor maior em X}
\PY{n}{ax}\PY{o}{.}\PY{n}{xaxis}\PY{o}{.}\PY{n}{set\PYZus{}minor\PYZus{}locator}\PY{p}{(}\PY{n}{plt}\PY{o}{.}\PY{n}{MultipleLocator}\PY{p}{(}\PY{l+m+mf}{0.2}\PY{p}{)}\PY{p}{)} \PY{c+c1}{\PYZsh{} divisor maior em X}
\PY{n}{ax}\PY{o}{.}\PY{n}{yaxis}\PY{o}{.}\PY{n}{set\PYZus{}major\PYZus{}locator}\PY{p}{(}\PY{n}{plt}\PY{o}{.}\PY{n}{MultipleLocator}\PY{p}{(}\PY{l+m+mf}{1.0}\PY{p}{)}\PY{p}{)} \PY{c+c1}{\PYZsh{} divisor maior em Y}
\PY{n}{ax}\PY{o}{.}\PY{n}{yaxis}\PY{o}{.}\PY{n}{set\PYZus{}minor\PYZus{}locator}\PY{p}{(}\PY{n}{plt}\PY{o}{.}\PY{n}{MultipleLocator}\PY{p}{(}\PY{l+m+mf}{0.1}\PY{p}{)}\PY{p}{)} \PY{c+c1}{\PYZsh{} divisor maior em Y}

\PY{c+c1}{\PYZsh{} propriedades das linhas}
\PY{n}{ax}\PY{o}{.}\PY{n}{grid}\PY{p}{(}\PY{n}{which}\PY{o}{=}\PY{l+s+s1}{\PYZsq{}}\PY{l+s+s1}{major}\PY{l+s+s1}{\PYZsq{}}\PY{p}{,} \PY{n}{axis}\PY{o}{=}\PY{l+s+s1}{\PYZsq{}}\PY{l+s+s1}{x}\PY{l+s+s1}{\PYZsq{}}\PY{p}{,} \PY{n}{linewidth}\PY{o}{=}\PY{l+m+mf}{0.75}\PY{p}{,} \PY{n}{linestyle}\PY{o}{=}\PY{l+s+s1}{\PYZsq{}}\PY{l+s+s1}{\PYZhy{}}\PY{l+s+s1}{\PYZsq{}}\PY{p}{,} \PY{n}{color}\PY{o}{=}\PY{l+s+s1}{\PYZsq{}}\PY{l+s+s1}{r}\PY{l+s+s1}{\PYZsq{}}\PY{p}{)}
\PY{n}{ax}\PY{o}{.}\PY{n}{grid}\PY{p}{(}\PY{n}{which}\PY{o}{=}\PY{l+s+s1}{\PYZsq{}}\PY{l+s+s1}{minor}\PY{l+s+s1}{\PYZsq{}}\PY{p}{,} \PY{n}{axis}\PY{o}{=}\PY{l+s+s1}{\PYZsq{}}\PY{l+s+s1}{x}\PY{l+s+s1}{\PYZsq{}}\PY{p}{,} \PY{n}{linewidth}\PY{o}{=}\PY{l+m+mf}{0.5}\PY{p}{,} \PY{n}{linestyle}\PY{o}{=}\PY{l+s+s1}{\PYZsq{}}\PY{l+s+s1}{:}\PY{l+s+s1}{\PYZsq{}}\PY{p}{,} \PY{n}{color}\PY{o}{=}\PY{l+s+s1}{\PYZsq{}}\PY{l+s+s1}{b}\PY{l+s+s1}{\PYZsq{}}\PY{p}{)}
\PY{n}{ax}\PY{o}{.}\PY{n}{grid}\PY{p}{(}\PY{n}{which}\PY{o}{=}\PY{l+s+s1}{\PYZsq{}}\PY{l+s+s1}{major}\PY{l+s+s1}{\PYZsq{}}\PY{p}{,} \PY{n}{axis}\PY{o}{=}\PY{l+s+s1}{\PYZsq{}}\PY{l+s+s1}{y}\PY{l+s+s1}{\PYZsq{}}\PY{p}{,} \PY{n}{linewidth}\PY{o}{=}\PY{l+m+mf}{0.75}\PY{p}{,} \PY{n}{linestyle}\PY{o}{=}\PY{l+s+s1}{\PYZsq{}}\PY{l+s+s1}{\PYZhy{}}\PY{l+s+s1}{\PYZsq{}}\PY{p}{,} \PY{n}{color}\PY{o}{=}\PY{l+s+s1}{\PYZsq{}}\PY{l+s+s1}{r}\PY{l+s+s1}{\PYZsq{}}\PY{p}{)}
\PY{n}{ax}\PY{o}{.}\PY{n}{grid}\PY{p}{(}\PY{n}{which}\PY{o}{=}\PY{l+s+s1}{\PYZsq{}}\PY{l+s+s1}{minor}\PY{l+s+s1}{\PYZsq{}}\PY{p}{,} \PY{n}{axis}\PY{o}{=}\PY{l+s+s1}{\PYZsq{}}\PY{l+s+s1}{y}\PY{l+s+s1}{\PYZsq{}}\PY{p}{,} \PY{n}{linewidth}\PY{o}{=}\PY{l+m+mf}{0.5}\PY{p}{,} \PY{n}{linestyle}\PY{o}{=}\PY{l+s+s1}{\PYZsq{}}\PY{l+s+s1}{:}\PY{l+s+s1}{\PYZsq{}}\PY{p}{,} \PY{n}{color}\PY{o}{=}\PY{l+s+s1}{\PYZsq{}}\PY{l+s+s1}{g}\PY{l+s+s1}{\PYZsq{}}\PY{p}{)}

\PY{c+c1}{\PYZsh{} para remover as ticks, adicione comentários}
\PY{c+c1}{\PYZsh{}ax.set\PYZus{}xticklabels([])}
\PY{c+c1}{\PYZsh{}ax.set\PYZus{}yticklabels([]);}

\PY{n}{plt}\PY{o}{.}\PY{n}{plot}\PY{p}{(}\PY{n}{x}\PY{p}{,}\PY{n}{x}\PY{p}{,}\PY{l+s+s1}{\PYZsq{}}\PY{l+s+s1}{k}\PY{l+s+s1}{\PYZsq{}}\PY{p}{)}
\PY{n}{plt}\PY{o}{.}\PY{n}{plot}\PY{p}{(}\PY{n}{x}\PY{p}{,}\PY{o}{\PYZhy{}}\PY{n}{x}\PY{o}{+}\PY{l+m+mi}{4}\PY{p}{,}\PY{l+s+s1}{\PYZsq{}}\PY{l+s+s1}{k}\PY{l+s+s1}{\PYZsq{}}\PY{p}{)}
\end{Verbatim}
\end{tcolorbox}

            \begin{tcolorbox}[breakable, boxrule=.5pt, size=fbox, pad at break*=1mm, opacityfill=0]
\prompt{Out}{outcolor}{30}{\hspace{3.5pt}}
\begin{Verbatim}[commandchars=\\\{\}]
[<matplotlib.lines.Line2D at 0x7fc4adc1b668>]
\end{Verbatim}
\end{tcolorbox}
        
    \begin{center}
    \adjustimage{max size={0.9\linewidth}{0.9\paperheight}}{04b-plotagem-matplotlib_files/04b-plotagem-matplotlib_61_1.png}
    \end{center}
    { \hspace*{\fill} \\}
    
    \hypertarget{plots-com-preenchimento}{%
\subsection{Plots com preenchimento}\label{plots-com-preenchimento}}

Podemos usar \texttt{fill\_between} para criar preenchimentos de área em
gráficos.

    \begin{tcolorbox}[breakable, size=fbox, boxrule=1pt, pad at break*=1mm,colback=cellbackground, colframe=cellborder]
\prompt{In}{incolor}{31}{\hspace{4pt}}
\begin{Verbatim}[commandchars=\\\{\}]
\PY{n}{x} \PY{o}{=} \PY{n}{np}\PY{o}{.}\PY{n}{linspace}\PY{p}{(}\PY{o}{\PYZhy{}}\PY{n}{np}\PY{o}{.}\PY{n}{pi}\PY{p}{,} \PY{n}{np}\PY{o}{.}\PY{n}{pi}\PY{p}{,} \PY{l+m+mi}{60}\PY{p}{)}
\PY{n}{y} \PY{o}{=} \PY{n}{np}\PY{o}{.}\PY{n}{sin}\PY{p}{(}\PY{l+m+mi}{2}\PY{o}{*}\PY{n}{x}\PY{p}{)}\PY{o}{*}\PY{n}{np}\PY{o}{.}\PY{n}{cos}\PY{p}{(}\PY{n}{x}\PY{o}{/}\PY{l+m+mi}{2}\PY{p}{)}

\PY{n}{plt}\PY{o}{.}\PY{n}{fill\PYZus{}between}\PY{p}{(}\PY{n}{x}\PY{p}{,}\PY{n}{y}\PY{p}{,}\PY{n}{alpha}\PY{o}{=}\PY{l+m+mf}{0.5}\PY{p}{)}\PY{p}{;}
\end{Verbatim}
\end{tcolorbox}

    \begin{center}
    \adjustimage{max size={0.9\linewidth}{0.9\paperheight}}{04b-plotagem-matplotlib_files/04b-plotagem-matplotlib_63_0.png}
    \end{center}
    { \hspace*{\fill} \\}
    
    \begin{tcolorbox}[breakable, size=fbox, boxrule=1pt, pad at break*=1mm,colback=cellbackground, colframe=cellborder]
\prompt{In}{incolor}{32}{\hspace{4pt}}
\begin{Verbatim}[commandchars=\\\{\}]
\PY{n}{x} \PY{o}{=} \PY{n}{np}\PY{o}{.}\PY{n}{linspace}\PY{p}{(}\PY{o}{\PYZhy{}}\PY{n}{np}\PY{o}{.}\PY{n}{pi}\PY{p}{,} \PY{n}{np}\PY{o}{.}\PY{n}{pi}\PY{p}{,} \PY{l+m+mi}{60}\PY{p}{)}
\PY{n}{f1} \PY{o}{=} \PY{n}{np}\PY{o}{.}\PY{n}{sin}\PY{p}{(}\PY{l+m+mi}{2}\PY{o}{*}\PY{n}{x}\PY{p}{)}
\PY{n}{f2} \PY{o}{=} \PY{l+m+mf}{0.5}\PY{o}{*}\PY{n}{np}\PY{o}{.}\PY{n}{sin}\PY{p}{(}\PY{l+m+mi}{2}\PY{o}{*}\PY{n}{x}\PY{p}{)}

\PY{n}{plt}\PY{o}{.}\PY{n}{plot}\PY{p}{(}\PY{n}{x}\PY{p}{,}\PY{n}{f1}\PY{p}{,}\PY{n}{c}\PY{o}{=}\PY{l+s+s1}{\PYZsq{}}\PY{l+s+s1}{r}\PY{l+s+s1}{\PYZsq{}}\PY{p}{)}\PY{p}{;}
\PY{n}{plt}\PY{o}{.}\PY{n}{plot}\PY{p}{(}\PY{n}{x}\PY{p}{,}\PY{n}{f2}\PY{p}{,}\PY{n}{c}\PY{o}{=}\PY{l+s+s1}{\PYZsq{}}\PY{l+s+s1}{k}\PY{l+s+s1}{\PYZsq{}}\PY{p}{)}\PY{p}{;}
\PY{n}{plt}\PY{o}{.}\PY{n}{fill\PYZus{}between}\PY{p}{(}\PY{n}{x}\PY{p}{,}\PY{n}{f1}\PY{p}{,}\PY{n}{f2}\PY{p}{,}\PY{n}{color}\PY{o}{=}\PY{l+s+s1}{\PYZsq{}}\PY{l+s+s1}{g}\PY{l+s+s1}{\PYZsq{}}\PY{p}{,}\PY{n}{alpha}\PY{o}{=}\PY{l+m+mf}{0.2}\PY{p}{)}\PY{p}{;}
\end{Verbatim}
\end{tcolorbox}

    \begin{center}
    \adjustimage{max size={0.9\linewidth}{0.9\paperheight}}{04b-plotagem-matplotlib_files/04b-plotagem-matplotlib_64_0.png}
    \end{center}
    { \hspace*{\fill} \\}
    

    % Add a bibliography block to the postdoc
    
    
    
    \end{document}
